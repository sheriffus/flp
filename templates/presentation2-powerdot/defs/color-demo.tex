%% color-package-demo.tex
%% Bent Petersen Oct 2000

\documentclass[11pt]{article}
%\usepackage[usenames]{color}
%\usepackage{color}
\usepackage[usenames,dvipsnames,pdftex]{color}

\title{\color{Blue}\jobname.tex}
\author{Bent Petersen}
\date{Oct 24, 2000}

\pagestyle{plain}
\setlength{\parindent}{0pt}
\setlength{\parskip}{2.0ex plus 0.6ex minus 0.4ex}

\begin{document}
\maketitle
\thispagestyle{empty}
The easiest way to use color text in \LaTeXe{} is to use the color
package. You load the color package in your preamble as usual
\begin{verbatim}
\documentclass[11pt]{article}
\usepackage[usenames]{color}
\end{verbatim}
Here the option [usenames] causes the definitions for the 68 colors
known to dvips (see next page) to be preloaded. Those colors can then
be specified by name. For example \verb={\color{BrickRed} BrickRed}=
produces {\color{BrickRed} BrickRed}. Alternately one can use the
command
\begin{verbatim}
\textcolor{BrickRed}{this is BrickRed}
\end{verbatim}
which produces \textcolor{BrickRed}{this is BrickRed}.
If the 68 predefined colors do not suffice, or if you do not want
to load the definitions of all of them, you may define your own
custom colors.
Here is a dark blue defined in terms of the RGB (red, green, blue) color model.
\begin{verbatim}
\definecolor{MyDarkBlue}{rgb}{0,0.08,0.45}
\begin{center}
{\color{MyDarkBlue}This color is MyDarkBlue}
\end{center}
\end{verbatim}
%%
\definecolor{MyDarkBlue}{rgb}{0.1,0,0.55}
\begin{center}
{\color{MyDarkBlue}This color is MyDarkBlue}
\end{center}
%%
Once we have defined the new color we can use it anywhere in
the current document. For example,
\verb=\textcolor{MyDarkBlue}{a nice blue}= produces
\textcolor{MyDarkBlue}{a nice blue}.
Custom colors may also be defined in terms of the CMYK
(cyan, magenta, yellow, black)
color model. Here is a light magenta.
%%
\begin{verbatim}
\definecolor{MyLightMagenta}{cmyk}{0.1,0.8,0,0.1}
\begin{center}
{\color{MyLightMagenta}This color is MyLightMagenta}
\end{center}
\end{verbatim}
%%
\definecolor{MyLightMagenta}{cmyk}{0.1,0.8,0,0.1}
\begin{center}
{\color{MyLightMagenta}This color is MyLightMagenta}
\end{center}

\begin{center}\large
The 68 standard colors known to dvips\vspace{0.5ex}
\begin{tabular}{|l|l|l|l|}
\hline
{\color{Apricot} Apricot}&
{\color{Aquamarine} Aquamarine}&
{\color{Bittersweet} Bittersweet}&
{\color{Black} Black}\\ \hline
{\color{Blue} Blue}&
{\color{BlueGreen} BlueGreen}&
{\color{BlueViolet} BlueViolet}&
{\color{BrickRed} BrickRed}\\ \hline
{\color{Brown} Brown}&
{\color{BurntOrange} BurntOrange}&
{\color{CadetBlue} CadetBlue}&
{\color{CarnationPink} CarnationPink}\\ \hline
{\color{Cerulean} Cerulean}&
{\color{CornflowerBlue} CornflowerBlue}&
{\color{Cyan} Cyan}&
{\color{Dandelion} Dandelion}\\ \hline
{\color{DarkOrchid} DarkOrchid}&
{\color{Emerald} Emerald}&
{\color{ForestGreen} ForestGreen}&
{\color{Fuchsia} Fuchsia}\\ \hline
{\color{Goldenrod} Goldenrod}&
{\color{Gray} Gray}&
{\color{Green} Green}&
{\color{GreenYellow} GreenYellow}\\ \hline
{\color{JungleGreen} JungleGreen}&
{\color{Lavender} Lavender}&
{\color{LimeGreen} LimeGreen}&
{\color{Magenta} Magenta}\\ \hline
{\color{Mahogany} Mahogany}&
{\color{Maroon} Maroon}&
{\color{Melon} Melon}&
{\color{MidnightBlue} MidnightBlue}\\ \hline
{\color{Mulberry} Mulberry}&
{\color{NavyBlue} NavyBlue}&
{\color{OliveGreen} OliveGreen}&
{\color{Orange} Orange}\\ \hline
{\color{OrangeRed} OrangeRed}&
{\color{Orchid} Orchid}&
{\color{Peach} Peach}&
{\color{Periwinkle} Periwinkle}\\ \hline
{\color{PineGreen} PineGreen}&
{\color{Plum} Plum}&
{\color{ProcessBlue} ProcessBlue}&
{\color{Purple} Purple}\\ \hline
{\color{RawSienna} RawSienna}&
{\color{Red} Red}&
{\color{RedOrange} RedOrange}&
{\color{RedViolet} RedViolet}\\ \hline
{\color{Rhodamine} Rhodamine}&
{\color{RoyalBlue} RoyalBlue}&
{\color{RoyalPurple} RoyalPurple}&
{\color{RubineRed} RubineRed}\\ \hline
{\color{Salmon} Salmon}&
{\color{SeaGreen} SeaGreen}&
{\color{Sepia} Sepia}&
{\color{SkyBlue} SkyBlue}\\ \hline
{\color{SpringGreen} SpringGreen}&
{\color{Tan} Tan}&
{\color{TealBlue} TealBlue}&
{\color{Thistle} Thistle}\\ \hline
{\color{Turquoise} Turquoise}&
{\color{Violet} Violet}&
{\color{VioletRed} VioletRed}&
{\color{White} White}\\ \hline
{\color{WildStrawberry} WildStrawberry}&
{\color{Yellow} Yellow}&
{\color{YellowGreen} YellowGreen}&
{\color{YellowOrange} YellowOrange}\\ \hline
\end{tabular}
\end{center}

\end{document} 
