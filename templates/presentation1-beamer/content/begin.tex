\title{Reasoning about Erlang values}
\subtitle{Verifying Lists}
\author{Cl\'{a}udio Amaral}
\institute {Chalmers University of Technology
           \\ and LIACC - Universidade do Porto
           }
\titlegraphic{}
\date{\today}
% --------------------------------------------------- Slide --
\begin{frame}[plain]
  \titlepage
  \begin{center} {
      \pgfuseimage{university-logo} \\ 
      
      \vspace{0.1cm}
      %\small Discussion leader: name, name University, city.
    }
  \end{center}
\end{frame}
% --------------------------------------------------- Slide --
\section[Contents]{}
% ------------------------------------------------------------
\begin{frame}
	\frametitle{Outline}
	\tableofcontents[%
%		currentsection, % causes all sections but the current to be shown in a semi-transparent way.
% 		currentsubsection, % causes all subsections but the current subsection in the current section to ...
% 		hideallsubsections, % causes all subsections to be hidden.
%		hideothersubsections, % causes the subsections of sections other than the current one to be hidden.
% 		part=, % part number causes the table of contents of part part number to be shown
%		pausesections, % causes a \pause command to be issued before each section. This is useful if you
% 		pausesubsections, %  causes a \pause command to be issued before each subsection.
% 		sections={ overlay specification },
	]
\end{frame}
