% -*- mode: latex; mode: folding -*-

\section{Predicate Specification}
\label{sec:pred-spec}

Here we describe our predicate specification language following some of
the principles in.
\todo{cite Yves Deville's ``logic programming''}
%
There are several ways to state a predicate's specification, we do not
declare our specification process to be superior to others in any way.
%
It is just a form to express some predicate features we thought to be of
importance in a way that suited our needs.


Our general specification form of a predicate \yap{p/n} consists, in its
core, of a set of specification clauses about input types.
%
\begin{code}
predicate specification clause
  p Id
parameter input types
  T1, ..., Tn
\end{code}
%
Various aspects of the predicate for the particular input type in
question can be added to a specification clause.
%
\begin{code}
parameter relation
  irel([A1,...,An])
mode
  i(Mi1,...,Min), [o(Mo1,...,Mon)]
range
  {Min,Max}
pre/post conditions
  Goal
\end{code}


\subsection{Types}

As mentioned in \refSec{sec:generators}, types in \plqc{} differ from
the regular notion of type.
%
Our types are not defined by a set of ground terms but rather the set of
terms produced by a generator.
%
These are the types used in predicate specification and are also
the only mandatory part of the specification.
%
The reason for this is the goal of the specification, which is to be
automatically tested.
%



The types mentioned in a predicate specification clause are \plqc{}
generators used to automatically create individual test cases.
%
The type in a specification clause is partial in the sense that it only
specifies the predicate should succeed when given elements of such types
as parameters.
%
It states nothing about other kinds of terms.



\subsection{Domain Precondition}
\subsection{Mode}
\subsection{Range}
\subsection{Properties}

