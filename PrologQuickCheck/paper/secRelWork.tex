% -*- mode: latex; mode: folding -*-

\section{Related Work} 
%% \paragraph{\bfseries \em Related Work} 
%% \subsubsection{Related Work} 
\label{sec:rel-work}


%% As far as we know, there seems to be no previous work on automatic
%% testing of \Prolog{} programs against specifications.%% , although tools
%
%% Although, 
{\sf SWI-Prolog}'s {\sf plunit}~\cite{swi-prolog} supports unit testing.
%
The {\sf Ciao Prolog System} ~\cite{ciao-prolog} has an assertion language
integrating run-time checking and unit testing~\cite{ciao-assert-unit}.
\todo{em vez de 'Prolog System' o comentario tinha 'LP xzxdf'}
%
We also use a property specification language but in an
automatic property-based randomly generated testing context.
%% We also use a property specification language but in an
%% automatic property-based randomly generated testing context.
%
Property specification languages for \Prolog{} were used
before~\cite{ciao-assert-unit,Deville1990logprog,somogyi1995mercury} in
different contexts.
%
However automated test is well supported in several languages and
paradigms.
%
The three most influential tools were \QuickCheck~\cite{quickcheck} for
the functional programming language \Haskell, {\sf PropEr}~\cite{papadakis2011proper} for %% the
%% functional programming language
{\sf Erlang}
and {\sf Curry}'s {\sf EasyCheck}~\cite{christiansen2008easycheck}.
%% and {\sf EasyCheck} \cite{christiansen2008easycheck}  for
%% the functional-logic language {\sf Curry}.


{\sf Easycheck} is an automated tool for specification-based testing of
declarative programs, which deals with logic programming features.
%
It is written in the functional-logic programming language Curry and it
is based on the view of free variables of an appropriate type as
nondeterministic generators~\cite{antoy2006overlapping} and mechanisms
to specify  properties of
nondeterministic operations by generalizing the set of combinators for
non-deterministic operations of \QuickCheck{}.
%
In our work we focus on \Prolog{} and, in contrast with {\sf EasyCheck}, non-deterministic generators are implemented by non-deterministic \Prolog{}
programs, types are implemented by monadic logic programs~\cite{fruhwirth1991logic,florido1992types}, and
we use a specification language for standard features of
%% with standard specification features of
logic programming such as modes and number of
answers~\cite{Deville1990logprog}.
%
%% {\sf EasyCheck} enables the use of different test-case generation strategies,
%% \plqc{} uses only depth-first test-case generation (although it is
%% not complete) because we want to exactly mimic \Prolog{} operational
%% semantics.


There are several automatic test tools for functional programming
languages, namely \QuickCheck, {\sf PropEr}, {\sf SmallCheck}~\cite{runciman2008smallcheck},
%% SparseCheck \cite{}, 
%% SML-QuickCheck \cite{...}, 
and G$\forall$ST~\cite{koopman2003gast}.
The %% first and
most preeminent tool is \QuickCheck{}.
%
\QuickCheck{} uses a domain specific language of testable specifications
as does {\sf PropEr}.
%
We define a specification language in \plqc{} but with
differences related to the relational nature of \Prolog{}.%% , such as modes
%% and constraints on the number of answers.
As in \QuickCheck{}, we also use random testing - we choose this method
compared to systematic methods due to its success in \QuickCheck{}.
%
%% The way
\QuickCheck{} generates test data %% depends
based
on \Haskell{} types.
In {\sf Erlang} types are dynamically checked and {\sf PropEr}, as well
as {\sf Erlang}\QuickCheck{}, guides value generation by functions,
being quantified types defined by these generating functions.
%
\Prolog{} is an untyped language, but type information is crucial also in
\plqc{} test data generation.
%
In parallel with {\sf Erlang} tools, we adopt the view of types as
monadic \Prolog{} programs describing sets of terms, widely used in the
{\em regular types} approach for typing logic programs~\cite{Yardeni:1991:TSL:110703.110705,fruhwirth1991logic,DBLP:conf/iclp/Zobel87,florido1992types}.


Recently a novel counterexample generator for the interactive theorem
prover Isabelle \cite{bulwahn2012smart} reformulates the preconditions
also as logic programs and analyzes them with a mode inference.
%
%% But t
The application of this tool is on the  synthesis of test data
generators for functional programming languages (e.g. ML, \Haskell) from
specifications in Isabelle, %% and is quite different from the use of logic
%% programs in \plqc{} generators,
but may be a promising approach for improving the generators.
\todo{posso ter interpretado mal o comentario para a alteração desta
  última frase}
%% The two most inspiring tools were QuickCheck \cite{...} for the
%% functional programming language Haskell and EasyCheck \cite{...}  for
%% the functional-logic language Curry.

%% Easycheck was another automated tool for specification-based testing of
%% declarative programs, which deals with logic programming features.
%% %
%% It was written in the functional-logic programming language Curry and it
%% is based on the view of free variables of an appropriate type as
%% nondeterministic generators \cite{OVERLAPPING RULES AND LOGIC
%%   VARIABLES...2006} and operations to specify properties of
%% nondeterministic operations by generalizing the set of combinators for
%% non-deterministic operations of QuickCheck.
%% %
%% In our work we focus on Prolog and, in contrast with EasyCheck, non
%% deterministic generators are implemented by non-deterministic Prolog
%% programs, types are implemented by monadic logic programs \cite{Types as
%%   Logic Programs - Fruewirth, Types as Theories, Florido and Damas}, and
%% we use a specification language with standard specification features of
%% logic programming such as modes and number of answers\cite{Yves
%%   Deville}.
%% EasyCheck enabled the use of different test-case generation strategies,
%% PrologCheck uses only depth-first test-case generation (although it is
%% not complete) because we want to exactly mimic Prolog operational
%% semantics.


%% There were several automatic test tools for functional programming
%% languages, namely QuickCheck \cite{...}, SmallCheck \cite{...},
%% SparseCheck \cite{...}, SML-QuickCheck \cite{...}, and G$\forall$ST
%% \cite{...}.
%% The first and most preminent tool was QuickCheck.
%% QuickCheck used a domain specific language of testable specifications.
%% We also define a specification language in PrologCheck but with
%% differences related to the relational nature of Prolog, such as modes
%% and constraints on the number of answers.
%% As in QuickCheck, we also use random testing - we choose this method
%% compared to systemmatic methods, due to its success in QuickCheck.
%% The way QuickCheck generated test data depends on Haskell types.
%% Prolog is an untyped language, but type information is crucial also in
%% PrologCheck test data generation.
%% We adopt the view of types as monadic Prolog programs describing sets of
%% terms, widely used in the  {\em regular types} approach for typing logic
%% programs \cite{RegularTypes - Yardeni,Shapiro, Logic Programs as Types
%%   for logic programs - Fruewirth, Types of Logic Programming - J. Zobel,
%%   Types and Theories - Florido and Damas}.
