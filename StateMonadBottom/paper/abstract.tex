Monads are ubiquitous in pure functional languages such as
Haskell.
\todo{CA: is this still true for pure functional languages in
  general? found at least one more (don't remember which). Dig into
  this.}
%
They are applied to structure programs and stateful computations where
state monads are extensively used. % ??
%% State monads are ubiquitous in Haskell.
%% %
The (state) monad laws enable reasoning with and refactoring of
(state) monadic code.
%
But do the laws really hold?
%
%% For finite, total values, the answer is yes, but Haskell has more
%% than that.
For finite, total values, the answer is yes, but in a setting with
lifted pairs and function spaces this is not the case.
%
We show that the implementations of the state monad in the Haskell
libraries fail to obey some interpretations of the laws.
%% %
%% In a setting with lifted pairs and function spaces we show that the
%% implementations of the state monad in the Haskell libraries fail to
%% obey some interpretations of the laws.


This paper explores possible interpretations of laws for the state
monad and implementations of state monads to check the laws against.
%
We suggest an interface of observable actions for implementations of
state monads together with a concrete meaningful set of laws that
account for the observability of state monadic expressions.
%
We also present an implementation of a state monad with such an
interface and that satisfies the mentioned laws.
