% --------------------------------------------------- Section --
\section{Search and Rescue}
% --------------------------------------------------- Slide ----
   \begin{frame}
       \frametitle{...}
       \tableofcontents[currentsection]
   \end{frame}
% --------------------------------------------------- SubSec ---
\subsection{(Some)}
%% % --------------------------------------------------- sSubSec --
%% \subsubsection{How inteligent?}
% --------------------------------------------------- Slide ----
\frame{
\frametitle{}

\begin{itemize}
  \item 
    \note{}
    \only<2>{
      \begin{itemize}
      \item theorem:\\
      \end{itemize}
    }
  \item<3-> 
    \only<1,3>{\uncover<4>{
      \begin{itemize}
      \item 1\\        2\\        3
      \end{itemize}
    }}
    

    
\end{itemize}

\only<4->{
  ``The writer [...] cannot help feeling, all the same, that the
  comparison reveals a fundamental inadequacy in their approach. There
  is no need to kill a chicken with a butcher’s knife. Yet the net
  impression is that Newell-Shore-Simon failed even to kill the chicken
  with their butcher’s knife.'' \cite{john_harrison:AR_short_survey}
  quoting Wang \cite{h_wang:to_mech_math_1960}
}


XXXXXXXXXXXXXXXXXXXXXXXXXXXXXXXXXXXXXXXXXXXXXXXXXXXXXXXXXXXXXXXXXXXXXX
0. W. W. Bledsoe. Some automatic proofs in analysis. In W. W. Bledsoe
and D. W. Loveland, editors, Automated Theorem Proving: After 25
Years, volume 29 of Contemporary Mathematics, pages 89–118. American
Mathematical Society, 1984.
XXXXXXXXXXXXXXXXXXXXXXXXXXXXXXXXXXXXXXXXXXXXXXXXXXXXXXXXXXXXXXXXXXXXXX
}



}


\frame{
\frametitle{}
 \alert<1>{automated} theorem
provers


}


\subsection{Spotting differences...}
\frame{
  \frametitle{}

  \vspace{-0.3cm}
}


