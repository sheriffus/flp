% -*- mode: latex; mode: folding -*-


\documentclass[11pt,twoside]{article}

\usepackage{techReps}
\usepackage{epsfig}

\usepackage{syntax}
\usepackage{bcprules,proof}

\parindent=0pt


% variables
\newcommand{\vf} {\mbox{\tt f}}
\newcommand{\vg} {\mbox{\tt g}}
\newcommand{\vs} {\mbox{\tt s}}
\newcommand{\vt} {\mbox{\tt t}}
\newcommand{\vtp} {\mbox{\tt t}^{\prime}}
\newcommand{\vtpp} {\mbox{\tt t}^{\prime\prime}}
\newcommand{\vu} {\mbox{\tt u}}
\newcommand{\vup} {\mbox{\tt u}^{\prime}}
\newcommand{\vv} {\mbox{\tt v}}
\newcommand{\vx} {\mbox{\tt x}}
\newcommand{\vy} {\mbox{\tt y}}
\newcommand{\vz} {\mbox{\tt z}}
% type variables
\newcommand{\vR} {\mbox{\tt R}}
\newcommand{\vS} {\mbox{\tt S}}
\newcommand{\vT} {\mbox{\tt T}}
\newcommand{\vTp} {\mbox{\tt T}^{\prime}}
\newcommand{\vV} {\mbox{\tt V}}
\newcommand{\vW} {\mbox{\tt W}}
% boolean constants
\newcommand{\ct} {\mbox{\tt true}}
\newcommand{\cf} {\mbox{\tt false}}
\newcommand{\cB} {\mbox{\tt Bool}}
% nat constants
\newcommand{\cz} {\mbox{\tt 0}}
\newcommand{\cN} {\mbox{\tt Nat}}
% term forms
\newcommand{\ite}[3] {\mbox{\tt if ${#1}$ then ${#2}$ else ${#3}$}}
\newcommand{\suc}[1] {\mbox{\tt succ ${#1}$}}  % succ is already defined in latex
\newcommand{\pred}[1] {\mbox{\tt pred ${#1}$}}
\newcommand{\iszero}[1] {\mbox{\tt iszero ${#1}$}}
\newcommand{\labs}[3] {\mbox{\tt $\lambda{#1}$:${#2}$.${#3}$}}
% judgments
\newcommand{\hast}[2] {{#1}:{#2}}
\newcommand{\tr}[3] {{#1}\vdash{#2}:{#3}}

% Prolog
\newcommand{\Prolog} {{\sf Logic Programming}}
% Cond Language
\newcommand{\Cond} {{\sf\small Cond}}

\newcommand{\PaperTitle} {
  \Prolog\ Axiomatic Semantics for Goals and Programs
}
\newcommand{\authorScholarship} {
  \thanks{author is funded by FCT grant SFRH/BD/XXXXX/20XX.}
}
\newcommand{\semTerm}[3] {
  \ensuremath{ \mathcal{T} [ #1 ] #2 #3}
}
\newcommand{\semAss}[3] {
  \ensuremath{ #1 \models^{#2} #3}
}




\makeatletter
\def\cleardoublepage{\clearpage\if@twoside \ifodd\c@page\else%
\hbox{}%
\thispagestyle{empty}
\newpage%
\if@twocolumn\hbox{}\newpage\fi\fi\fi}
\makeatother

\begin{document}

\trtitle{\PaperTitle}
\trauthor{Cl\'audio Amaral \hspace{1cm} M\'ario Florido
  \hspace{1cm} V\'itor Santos Costa}

\trnumber{DCC-2014-XX\\Version $1.0$ September 2014}
%% \trnumber{DCC-2006-07\\Version $1.1$ September 2013}

\mkcoverpage

\title{Axiomatic Semantics for \Prolog}
\author{Cl\'audio Amaral\authorScholarship \hspace{0.5cm}
 M\'ario Florido \hspace{0.5cm}
V\'itor Santos Costa
\\ {\tt \{coa,amf,vsc\}@dcc.fc.up.pt}}

\date{}
\cleardoublepage

\maketitle

\abstract

Axiomatic semantics represent an important role in the verification and
certification of high quality software.
%
They enable the derivation of properties of programs statically and are
a first step into automated verification of programs.


Since their introduction, axiomatic semantic systems have been designed
for many programming languages, most of which imperative languages.


In this paper we present an axiomatic semantic system for \Prolog.
%
We provide a soundness proof of the rules w.r.t. a previously defined
operational semantics~\cite{PlOpSem}.





\section{Introduction}
\label{sec:intro}

%{{{ what

\subsection{what}

Axiomatic Semantics is a method to proving the correctness of
programs.
%
The approach works by expressing the semantics of a programming language
through a formal logic system coupled to the language in which program
properties can be proven.


%}}}

%{{{ why
\subsection{why}

Although the use of Axiomatic Semantics is widespread, they do not cover
all programming paradigms evenly.


%}}}

%{{{ how
\subsection{how}

We do stuff

%}}}

%{{{ outline
\subsection{Paper outline}
This paper is organised as follows
%}}}


\section{Related Work}
\label{sec:rel-work}
%{{{

Java, C, Winskel book, Prolog Axiomatic Semantics from the 90s.

%}}}

\section{Semantics of Assertions}
\label{sec:sem-ass}

\subsection{The Assertion Language}

\paragraph{\Cond\ Language}
%{{{

\setlength{\grammarparsep}{20pt plus 1pt minus 1pt} % increase separation between rules
\setlength{\grammarindent}{5em} % increase separation between LHS/RHS 

\begin{grammar}
<\Cond> ::= `true'\ \ |\ \ `false'
\ |\ \  <term> `=' <term> 
\alt  <term> `<=' <term>
\ \ |\ \  $\neg$ <\Cond>
\alt <\Cond> $\wedge$ <\Cond>
\ \ |\ \  <\Cond> $\vee$ <\Cond>
\alt  $\forall$ <var>`.' <\Cond>
\ \ |\ \  $\exists$ <var>`.' <\Cond>

<term> ::= <var> \ \ |\ \ <symbol> (`('<terms>`)')?

<terms> ::= <term> \ \ |\ \ <term> `,' <terms>


<var> ::= [A-Z][a-zA-Z]*
\end{grammar}


It is important to note that terms in \Cond{} are implicitly interpreted
from to program to the logic language.
%
Integers in \Cond{} are a subset of terms where the interpretation is
the representation of a number.


\Cond{} is extended with logical predicates on terms, which include modes
and types, which can be viewed as meta-predicates of the program.

\begin{grammar}
<\Cond> ::= ...
\alt  <pred> `(' <term> `,' ... `,' <term>`)'
\end{grammar}

Without loss of generality we assume \Cond{} has integer arithmetic and
terms can be integers

\begin{grammar}
<i> ::= <num>
\ \ |\ \ <var>
\ \ |\ \ <const>
\ \ |\ \ <i> <op> <i>

<num> ::= \dots
<var> ::= \dots
<const> ::= \dots

<op> ::= `+'
\ \ |\ \ `-'
\ \ |\ \ `*'
\end{grammar}

Constants may be defined in the interpretation used.


$FV(t)$ -- free variables of $t$

$t \sigma$ -- the result of applying the substitution $\sigma$ to term $t$

%}}}

\subsection{Semantics of terms and atoms}

%{{{

\begin{eqnarray*}
\semTerm{f(t_1,...,t_n)}{I}{\sigma}& =& 
I(f)(\semTerm{t_1}{I}{\sigma}, \ldots, \semTerm{t_n}{I}{\sigma})
\\
\semTerm{x}{I}{\sigma}& =& \sigma(x)
\\
\semTerm{P(t_1,...,t_n)}{I}{\sigma}& =& 
I(P)(\semTerm{t_1}{I}{\sigma}, \ldots, \semTerm{t_n}{I}{\sigma})
\end{eqnarray*}

%}}}


\subsection{Semantics of assertions}

%{{{

\begin{align*}
\semAss{\sigma}{I}{&true}
 \\
\semAss{\sigma}{I}{&P(t_1,...,t_n)}
 && \text{iff } \semTerm{P(t_1,...,t_n)}{I}{\sigma}
\\
\semAss{\sigma}{I}{&A \wedge B}
 &&\text{iff } \semAss{\sigma}{I}{A} \text{ and } \semAss{\sigma}{I}{B}
\\
\semAss{\sigma}{I}{&A \vee B}
 &&\text{iff } \semAss{\sigma}{I}{A} \text{ or } \semAss{\sigma}{I}{B}
\\
\semAss{\sigma}{I}{&\neg A }
 &&\text{iff } \text{not } \semAss{\sigma}{I}{B}
\\
\semAss{\sigma}{I}{&A \Rightarrow B}
 &&\text{iff not }( \semAss{\sigma}{I}{A} )\text{ or } \semAss{\sigma}{I}{B}
\\
\semAss{\sigma}{I}{&\forall x. A}
 &&\text{iff } \semAss{\sigma[t/x]}{I}{A} \text{ for all } t \in \Terms
\\
\semAss{\sigma}{I}{&\exists x. A}
 &&\text{iff } \semAss{\sigma[t/x]}{I}{A} \text{ for some } t \in \Terms
\\
\semAss{\bot}{I}{&A}
 &&
\end{align*}

The semantics of a predicate when not defined in $I$ is its operational
semantics.

%}}}

\subsection{Extension of an assertion}

Extension of $A$, an assertion, w.r.t. an interpretation $I$

$$ A^{I} = \{ \sigma \in \text{Subst}_{\bot}\ |\ \semAss{\sigma}{I}{A} \} $$

\subsection{Partial correctness assertions}

$$ \langle \text{code}, [\langle A_{1}, B_{1} \rangle,\ldots, \langle
  A_{n}, B_{n} \rangle] \rangle $$

satisfaction relation between stated and partial correctness assertions,
w.r.t. $I$

\begin{align*}
&\semAss{\sigma}{I}{\langle \text{code}, [\langle A_{1}, B_{1}
      \rangle,\ldots, \langle A_{n}, B_{n} \rangle] \rangle}
\\iff&\\
&\semAss{\sigma}{I}{A_{i}} \Rightarrow
 \semAss{\semLogProg{\text{code}}{\sigma}{I}_{i}}{I}{B_{i}}
\end{align*}
913 309 015
\subsection{Validity}
$$\forall \sigma \in \text{Subst}_{\bot}. \semAss{\sigma}{I}{\langle \text{code}, [\langle A_{1}, B_{1}
      \rangle,\ldots, \langle A_{n}, B_{n} \rangle] \rangle}$$

we can write as 

$$ \semAss{}{I}{\langle \text{code}, [\langle A_{1}, B_{1}
      \rangle,\ldots, \langle A_{n}, B_{n} \rangle] \rangle}$$

expressing the partial correctness assertion is valid
w.r.t. interpretation $I$

We are interested in whether or not it is true for all stated and all
interpretations that encompass $I_{mathcal{PL}}$ ($\forall
I. I_{\mathcal{PL}} \sqsubseteq I$).
%
This motivates the notion of \underline{validity}

$$ \semAss{}{}{\langle \text{code}, [\langle A_{1}, B_{1}
      \rangle,\ldots, \langle A_{n}, B_{n} \rangle] \rangle}$$

$$ \equiv \forall I _{I_{\mathcal{PL}} \sqsubseteq I}. \forall \sigma.
\semAss{\sigma}{I}{\langle \text{code}, [\langle A_{1}, B_{1}
      \rangle,\ldots, \langle A_{n}, B_{n} \rangle] \rangle}$$

\section{Axiomatic Semantics}
\label{sec:ax-sem}
The Axiomatic Semantics presented in this paper is a formal system for
proving properties of logic programs.
%
The system is based on proof rules defined for each programming
construct.
%
The proof rules are inspired by Hoare rules~\cite{} for imperative
languages, which evolved from Floyd's work~\cite{}.


%{{{ notation

$$      Code \qquad  spec\\
      \\
      spec :: [ (Cond, Cond)]\\
      $$an ordered collection of pairs where the first element is a
      pre-condition and the second element is a corresponding post-conditions$$
      \linebreak
      \hookleftarrow  $$-- notation for retrieving each element of an
      ordered collection$$
      \linebreak
      \diamond\  $$-- notation for the concatenation of
      ordered collections$$
      \linebreak
      f\|\ cs $$-- notation for the application of a function
      ordered collections$$
      \linebreak
      f\bigcirc\ cs $$-- notation for the application of a function
      returning an ordered collection to an
      ordered collections (concatenation of all partial results)$$
      \linebreak
      id $$-- the identity collection, compatible with all specific
      element orderings$$
      \linebreak
      \sigma \models^I Cond  $$-- state/substitution$$\ \sigma\ $$satisfies
      Cond in interpretation I, or equivalently, that Cond is true at
      state/subst$$\ \sigma$$, in interpretation I$$
      \linebreak
      \perp \models Cond\ $$-- for any Cond (needed when
      $$\mathcal{S}_{op}[|code|]\sigma\ $$is undefined)$$
      \linebreak
      Cond ::= true | false | t_1 = t_2 | C \wedge C | C \vee C | \neg c |
      C \Rightarrow C | \forall t. C | \exists t. C | Pred(t)
      \linebreak
      Pred(t)\ $$-- describes computable properties os a predicate, such
      as types and modes$$
      \linebreak
      \forall t. type(t) \Rightarrow C \equiv \forall t \in type. C
      \linebreak
      \mathcal{S}_{op}[|goal|]\sigma :=
      interp(\langle \langle goal^f,  \rangle::nil, \sigma, nil \rangle
      :: nil, \mathcal{Prog})
      \linebreak
       (g_1 \wedge g_2)^f ::= g_1^f \diamond g_2^f
      \linebreak
       (atom(V_1,\ldots,V_n))^f ::= [ \langle atom(V_1,\ldots,V_n),
        [V_1,\ldots,V_n] \rangle ]
      \linebreak
      \sigma \models Cond[t/x] \equiv \sigma[t\sigma/x] \models Cond
$$

%}}}

\subsection{Proof Rules}

%{{{ proof rules

\infrule[Pred]
        {\PCAAx{\Pair{\Clause{h}{b}}{\spec{s}}}
          {\Goal{b}}{\spec{s}}}
        {\PCAA{\Clause{h}{b}}{\spec{s}}}

\infrule[Disj]
      {\PCAA{\Goal{g$_1$}}{\spec{s$_1$}} \andalso
        \PCAA{\Goal{g$_2$}}{\spec{s$_2$}}}
      {\PCAA{\Goal{\LgOr{g$_1$}{g$_2$}}}{\Concat{s$_1$}{s$_2$}}}
      
\infrule[Conj]
      {\PCAA{\Goal{g$_1$}}{\spec{s$_1$}} \andalso
        \PCAA{\Goal{g$_2$}}{\spec{s$_2$}}}
      {\PCAA{\Goal{\LgAnd{g$_1$}{g$_2$}}}{\spec{s$_{3}$}}}
\vspace{-18pt}
\infax{
 {\mbox{\tt s$_{3} =$
$[$\Pair{c$_{1A}$}{c$_{2B}$}$|$\Pair{c$_{1A}$}{c$_{1B}$}$\hookleftarrow$s$_1$,
 }}}
\vspace{-20pt}
\infax{
 {\mbox{\tt
\Pair{c$_{2A}$}{c$_{2B}$}$\hookleftarrow$s$_2$, c$_{1A} = $c$_{1B} ]$
 }}}


\infrule[Call]
  {\PCAA{\Clause{h(\textoverline{X})}{b}}{\spec{s}}}
  {\PCAA{\Goal{h(\textoverline{Y})}}
    {\spec{s[Y$_{1}$/X$_{1}$, $\ldots$, Y$_{n}$/X$_{n}$]}}}


\infrule[Cons]
      {\Pair{\Goal{g}}{\spec{s}}}
      {\Pair{\Goal{g}}{\spec{s$_{1}$}}}
\vspace{-18pt}
\infax{
 {\mbox{\tt s$_{1} =$
$[$\Pair{c$_{A}'$}{c$_{B}'$}$|$\Pair{c$_{A}$}{c$_{B}$}$\hookleftarrow$s,
 }}}
\vspace{-20pt}
\infax{ 
 {\mbox{\tt
$\exists $c$_{A}'$. c$_{A}$ $\Rightarrow$ c$_{A}'$,
$\exists $c$_{B}'$. c$_{B}'$ $\Rightarrow$ c$_{B} ]$
 }}}

\infax[UVar]
  {\PCAA{\Goal{$x=t$}}{\spec{[\Pair{C$_{A}$[t/x]}{\LgAnd{C$_{A}$}{$x=t$}}]}}}

\infax[URefl]
  {\PCAA{\Goal{$t=t$}}{\spec{[\Pair{C$_{A}$}{C$_{A}$}]}}}

\infrule[URefl]
  {\PCAA{\Goal{$x=t$}}{\spec{s}}}
  {\PCAA{\Goal{$t=x$}}{\spec{s}}}

\infrule[UFun]
  {\PCAA{\Goal{$t_{11}=t_{21} \wedge \ldots \wedge t_{1n}=t_{2n}$}}{\spec{s}}}
  {\PCAA{\Goal{$f(t_{11} \wedge \ldots \wedge t_{1n})=f(t_{21} \wedge \ldots \wedge t_{2n})$}}{\spec{s}}}


\eject

%}}}

some text

\subsection{Examples}
Here is the derivation of a property of a recursive  implementation of factorial.

\begin{sidewaysfigure}
%\begin{figure*}
\mbox{\tt
f(X,F) $\leftarrow$ body(f(X,F))
}
where\ \ 
\mbox{\tt
body(X,F) ::= $\leftarrow$ (X==0 $\wedge$ F is 1)
     $\vee$  (X>0 $\wedge$ X1 is X-1 $\wedge$ f(X1,F1) $\wedge$ F is F1*X).
}

\caption{Recursive factorial logic implementation.}
\end{sidewaysfigure}
%\end{figure*}

\begin{sidewaysfigure}
%% \begin{figure*}
%\vspace*{0.5in}


\infer[\mbox{\sc Call}]
 { \PCAe{\Goal{f(N,R)}}{\spec{[\ssNRakPz, \ssNRbkPz]}} }
 { \infer[\mbox{\sc Pred}]
  { \PCAe{\Clause{f(X,F)}{ body(f(X,F)) }} {\spec{[\ssXFakPz, \ssXFakPz]}} }
  { \infer[\mbox{\sc Disj}]
    { \PCA
      { \Pair{\Clause{f(X,F)}{ body(f(X,F)) }}{\spec{[\ssXFakPz, \ssXFbkPz]}}}
      { \Goal{body(f(X,F))}} {\spec{[\ssXFakPz, \ssXFbkPz]}}  }
    { \infer[\mbox{\sc Conj}]
      { \Pair{\Goal{\LgAnd{X==0}{F is 1}}}{\spec{[\ssXFakaPz]}} }
      { \infer[\mbox{\sc Cons}]
        { \Pair{\Goal{X==0}} {\spec{[\ssXFakPaka]} }}
        { \Pair{\Goal{X==0}} {\spec{[\ssXFakaPaka]} }
&
          \infer[\mbox{\sc Conj}]
            { a }
            { aa }
        }
&
    \infer[\mbox{\sc rule}]
    { \PCA{...}
            { \Pair{\Goal{F is 1}} {\spec{[\ssXFakaPz]} } }
            {b} }
    { \PCA{...}{a}{b} }
}}}}

\vspace*{0.5in}
where
\mbox{$mathcal{A} = $ }
\caption{Property derivation for a recursive factorial.}
\end{sidewaysfigure}
%% \end{figure*}

\subsection{Soundness}

``Every rule should preserve validity, in the sense that if the
assumptions in the rule's premise are valid then so is its conclusion.''
``When this holds of a rule it is called {\it sound}.''
``When every rule of a proof system is sound, the proof system itself is
said to be {\it sound}.''
``It follows then by rule-induction that every theorem obtained from the
proof system [of Hoare rules] is a valid partial correctness
assertion.''




If we can show each rule is sound, (preserves validity in the sense that
if its premises consists of valid assertions and partial correctness
assertions then so its conclusion).

\section{Axiomatic Semantics without Cons}
\label{sec:ax-sem-wo-cons}
\subsection{Proof Rules}



\subsection{Equivalence}

\section{Verification Conditions}
\label{sec:ver-cond}
\subsection{Weakest Precondition}
\subsection{VCGen}

\section{Logic Programming Verification Arquitecture}
\label{sec:ver-arch}
Small description of the whole verification process, from source to
proof.

Logic Program\ $+$ Assertions
 $\Rightarrow$ Axiomatic Semantics VCGen
 $\Rightarrow$ Proof Obligations
 $\Rightarrow$ Why3
 $\Rightarrow$ Provers



\section{Package Usage Examples}

I use an ad-hoc collection of tools for typesetting:
\begin{itemize}
\item
For representing object language syntax, I define macros in
the style shown above, extending the set as necessary to cover
new language features. This is tedious, but it helps me keep
things well-formed.
\item 
For stating inference rules, I use the {\tt bcprules} package.
\item
For showing derivations, I use the {\tt proof} package.
\end{itemize}


For example, here are the inference rules in Figures 8-1 and 8-2 on p.93:

\infax[T-True]{\hast{\ct}{\cB}}

\infax[T-False]{\hast{\cf}{\cB}}

\infrule[T-If]
  {\hast{\vt_1}{\cB} \andalso \hast{\vt_2}{\vT} \andalso \hast{\vt_3}{\vT}}
  {\hast{\ite{\vt_1}{\vt_2}{\vt_3}}{\vT}}

\infax[T-Zero]
  {\hast{\cz}{\cN}}

\infrule[T-Succ]
  {\hast{\vt_1}{\cN}}
  {\hast{\suc{\vt_1}}{\cN}}

\infrule[T-Pred]
  {\hast{\vt_1}{\cN}}
  {\hast{\pred{\vt_1}}{\cN}}

\infrule[T-IsZero]
  {\hast{\vt_1}{\cN}}
  {\hast{\iszero{\vt_1}}{\cB}}

\eject
Here is the text of Lemma 8.2.2 on p. 94:

\begin{enumerate}
\item
If $\hast{\ct}{\vR}$, then $\vR = \cB$.
\item
If $\hast{\cf}{\vR}$, then $\vR = \cB$.
\item
If $\hast{\ite{\vt_1}{\vt_2}{\vt_3}}{\vR}$, then 
$\hast{\vt_1}{\cB}$,$\hast{\vt_2}{\vR}$, and $\hast{\vt_3}{\vR}$.
\item
If $\hast{\cz}{\vR}$, then $\vR = \cN$.
\item
If $\hast{\suc{\vt_1}}{\vR}$, then $\vR = \cN$ and $\hast{\vt_1}{\cN}$.
\item
If $\hast{\pred{\vt_1}}{\vR}$, then $\vR = \cN$ and $\hast{\vt_1}{\cN}$.
\item
If $\hast{\iszero{\vt_1}}{\vR}$, then $\vR = \cB$ and $\hast{\vt_1}{\cN}$.
\end{enumerate}


Finally, here is the derivation near the bottom of p. 94:

\vspace*{0.5in}

\infer[\mbox{\sc T-If}]
{\hast{\ite{\iszero{\cz}}{\cz}{\pred{\cz}}}{\cN}}
{
 \infer[\mbox{\sc T-IsZero}]
 {\hast{\iszero{\cz}}{\cB}}
 {\infer[\mbox{\sc T-Zero}]
   {\hast{\cz}{\cN}} 
   {}}
 &
 \infer[\mbox{\sc T-Zero}]
  {\hast{\cz}{\cN}}
  {}
 &
 \infer[\mbox{\sc T-Pred}]
 {\hast{\pred{\cz}}{\cN}}
 {\infer[\mbox{\sc T-Zero}]
   {\hast{\cz}{\cN}}
   {}}
}

\vspace*{0.5in}

I'm sure there are better ways of doing some or all of these things;
contributions and suggestions are welcome!


\section*{Acknowledgment}
\label{sec:ack}


The authors would like to thank...

\section*{Conclusion}
\label{sec:conc}



\bibliographystyle{abbrv}
\bibliography{../plas}


\end{document}


