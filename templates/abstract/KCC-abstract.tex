\documentclass[11pt]{llncs}

\usepackage[english]{babel}
\usepackage{amsfonts,amssymb,bbm,mdwlist}
\usepackage[ansinew]{inputenc}
\usepackage{a4wide}
\newcounter{dummy}
\usepackage{color}

\newcommand{\here}{\textcolor{red}{$\bullet$\ }}

\renewcommand{\theequation}{\thesection.\arabic{equation}}
\renewcommand{\theproposition}{\thesection.\arabic{proposition}}
\renewcommand{\thecorollary}{\thesection.\arabic{corollary}}
\renewcommand{\thelemma}{\thesection.\arabic{lemma}}

\setlength{\parindent}{0mm}
\def\m{{\bf m}}
\def\ds{\displaystyle}
\def\nn{\mathbb N}
\def\rr{\mathbb R}
\def\cqd{$\Box$}





\title{Kolmogorov Complexity Cores}



\author{
Andr\'{e} Souto \thanks{\email{andresouto@dcc.fc.up.pt}}\\
}

\institute{Universidade Porto
\\
Instituto de Telecomunica\c{c}\~{o}es
}

\begin{document}

\maketitle

\begin{abstract}
A polynomial complexity core of a language is the set of ``hard
instances for any machine deciding the language. Computational depth
measures the useful information contained in a string as the
difference between the time bounded Kolmogorov complexity and
traditional Kolmogorov complexity. We establish a connection between
the existence of polynomial complexity cores of a set and the
computational depth of its characteristic sequence. In particular, we
show that if the computational depth of the characteristic sequence of
a set is larger than a constant (resp. any polynomial) almost
everywhere, then it has a (resp. proper) polynomial complexity
core. Finally, we study the average case complexity of
these sets proving that, if they have a polynomial complexity core with
exponential
density then they can not be recognizable in average polynomial time
when a time bounded
version of the universal distribution is considered.

\end{abstract} 
 
 
 
\end{document} 
 
 
