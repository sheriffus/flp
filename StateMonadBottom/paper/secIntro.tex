% -*- mode: latex; mode: folding -*-

\section{Introduction}
\label{sec:intro}
%{{{ introduction text

%Narrative: 
%  What (is the subject),
%  why  (are we exploring it) and 
%  how  (are we doing it)

%Start of "What" 
Monads were introduced by Moggi~\cite{Moggi} to the computer science
world  as a way to structure the semantics of effectful functional
computations.
%
Wadler\cite{MFFP} further popularised them in the functional programming
community and they have become very closely associated with Haskell
\cite{haskell98}.


State monads are widely used in pure functional programs for
structuring stateful computations.
%
The (state) monad laws explain how to refactor (state) monadic code.
%
Refactoring is used for calculating programmes from specifications and
compilers \todo{cite GHC} use refactoring to optimise the code.
%
The correctness of the refactoring steps depends on the (state) monad
laws correctness.


%  introducing "why"
But do the laws really hold?
%
For finite, total values, the answer is yes, but in a setting with
lifted pairs and function spaces this is not the case.
%
We show that the implementations of the state monad in the Haskell
libraries fail to obey some interpretations of the laws.
%% For finite, total values, the answer is yes, but Haskell has more than
%% that.
%% %
%% In a setting with lifted pairs and function spaces we show that the
%% current standard implementations of the sate monad do not satisfy its
%% laws.


%  how  (are we doing it)
This paper explores possible implementations of state monad instances
that may better satisfy the intended laws.
%
These implementations will be matched with either a counter-example or
a proof of its correctness w.r.t. the laws in a small language we
define in \refSec{sec:language}.
%
We perform QuickCheck~\cite{claessen-hughes-quickcheck} property (law)
based testing with the
ClassLaws~\cite{jeuring-jansson-amaral:2012:classlaws} framework to
help find counter-examples of Haskell state monad implementations
which we translate to our language.
%
In order to prove correct an implementation w.r.t. the laws we apply
equational reasoning.
%
It is also discussed how to interpret the laws of the state monad and
given an adequate notion of observable equality to evaluate the laws
with.
%
We give an implementation of a state monad and proof that it satisfies
this interpretation of the laws.

%}}} introduction


%{{{ paper outline
This paper is organised as follows.
%
\refSecI{sec:stm} introduces Haskell versions of state monad
implementations we intend to check.
%
\refSecI{sec:equalities} shows two different equality checks for
(state) monadic computations and counter-examples for the laws using
each equality for each implementation.
%
\refSecI{sec:language} introduces a small pure functional call-by-need
language with which we reproduce the implementations in
\refSec{sec:stm} and go through the respective counter-examples for
the equalities in \refSec{sec:equalities}.
%
\refSecI{sec:obs-eq} defines an observable equality check for state
monadic computations and shows an implementation satisfying the laws
given observable equality over state monadic terms.
%
\refSecI{sec:conclusion} concludes with a discussion of the results
and possible future work.
%}}} outline


%% \subsection{Preliminaries}
%{{{ some preliminary information
%% \paragraph{}{Testing will be property-based and/or exhaustive.
%% %
%% Property-based testing will be done using Haskell
%% QuickCheck \cite{claessen-hughes-quickcheck} and exhaustive testing
%% will require the design and implementation of some domain
%% enumeration techniques.}

%% \paragraph{}{Bottom values shall be addressed relying on the
%%   ChasingBottoms Haskell library 
%%   ({\tt \$: cabal install ChasingBottoms}) introduced in
%%   \cite{danielssonjansson04:chasingbottoms}.
%% }

%% \paragraph{}{Reasoning about the State Monad implementations can
%%   follow the procedure in \cite{danielssonetal06:fastandloose} and try
%%   to reach conclusions about totality$\times$strictness.
%% }

%% \paragraph{}{Monomorphisation of the State Monad type parameters
%%   should follow a technique/theory with some guaranties, as in
%%   \cite{bernardy_testing_2010} (other approaches should be checked as
%%   well).
%% }
%}}} some preliminary information


