\documentclass[
mode=present,      % fazer coisas para apresenta��o
paper=screen,      % a fazer num ecr�
orient=landscape,  % com esta orienta��o
%orient=portrait,   % com esta orienta��o
display=slides,
style=aggie,
blackslide,
%clock
usenames,dvipsnames,
]{powerdot}

\usepackage[latin1]{inputenc}
%\usepackage[portuges]{babel}
\usepackage[english]{babel}

% \usepackage{color}
% \usepackage{amsfonts}
\usepackage{amsmath}
\usepackage{amssymb}
\usepackage{amsthm}
% \usepackage{xspace}
% \usepackage{algorithm}
% \usepackage{algorithmic}
% \usepackage{epsf}
% \usepackage{graphicx}
% \usepackage{enumerate}
% \usepackage{alltt}
% \usepackage[pstricks1-10]{vaucanson-g}
% \usepackage{vaucanson-g}
% \usepackage{supertabular}

\usepackage{graphicx}

%\usepackage[usenames]{color}
\usepackage{color}
%\usepackage[usenames,dvipsnames,pdftex]{color}
%\usepackage[dvipsnames]{color}

\newcommand{\whyReserved}[1]{{\texttt{\bfseries\color{Cerulean}#1\xspace}}}
\newcommand{\whyTrue}{{\texttt{\bfseries\color{Cerulean}true\xspace}}}
\newcommand{\whyFalse}{{\texttt{\bfseries\color{Cerulean}false\xspace}}}
\newcommand{\whyAnd}{{\texttt{\bfseries\color{Cerulean}and\xspace}}}
\newcommand{\whyOr}{{\texttt{\bfseries\color{Cerulean}or\xspace}}}
\newcommand{\whyIf}{{\texttt{\bfseries\color{Cerulean}if\xspace}}}
\newcommand{\whyThen}{{\texttt{\bfseries\color{Cerulean}then\xspace}}}
\newcommand{\whyElse}{{\texttt{\bfseries\color{Cerulean}else\xspace}}}
\newcommand{\whyLet}{{\texttt{\bfseries\color{Cerulean}let\xspace}}}
\newcommand{\whyIn}{{\texttt{\bfseries\color{Cerulean}in\xspace}}}
\newcommand{\whyType}[1]{{\texttt{\bfseries\color{LimeGreen}#1\xspace}}}
\newcommand{\whyFunction}[1]{{\texttt{\bfseries\color{Black}#1}}}
%\newcommand{\whyFunction}[1]{{\texttt{\bfseries\color{MidnightBlue}#1}}}
\newcommand{\whyComment}[1]{{\texttt{\bfseries\color{BrickRed}#1\xspace}}}
\newcommand{\whyVar}[1]{{\texttt{\bfseries\color{Dandelion}#1\xspace}}}
\newcommand{\whyOther}[1]{{\texttt{\bfseries\color{Black}#1\xspace}}}
%\newcommand{\type}{\colouring{Turquoise}type\xspace}
\newcommand{\whyOP}{{\texttt{\bfseries\color{Cerulean}\,$($\,\xspace}}}
\newcommand{\whyCP}{{\texttt{\bfseries\color{Cerulean}\,$)$\,\xspace}}}
\newcommand{\whyColon}{{\texttt{\bfseries\color{Cerulean}$:$\xspace\,}}}
\newcommand{\whyArrow}{{\texttt{\bfseries\color{Cerulean}\,$\rightarrow$\xspace\,}}}
\newcommand{\whyEq}{{\texttt{\bfseries\color{Cerulean}\,$=$\,\xspace}}}
\newcommand{\whyDot}{{\texttt{\bfseries\color{Cerulean}\, .\,\xspace}}}


\newtheorem*{definition}{Definition}
\newtheorem*{example}{Example}

\title{ Towards the Verification of Erlang Code:\\Data Representation }
% \subtitle{ Data Representation }
\author{ Cl\'audio Amaral }
% \email{coa@ncc.up.pt}
% \institution{Laborat�rio de Intelig�ncia Artificial e Ci�ncia de
% Computadores - Universidade do Porto}
\date{28 January 2010}

\begin{document}

\maketitle                                   %%% = slide 1

\begin{slide}{Forecast}                      %%% = slide 2
\begin{figure}[ht]
\centering
\includegraphics[width=0.8\textwidth]{wtf.eps}
\end{figure} 

\end{slide}


\begin{slide}{Outline}                       %%% = slide 3

\onslide*{1-}{
  \begin{itemize}[type=1]
    \item<1-> $<self>$
    \item<1-> Introduction
      \begin{itemize}[type=1]
        \item<1-> Context
        \item<1-> Objective
      \end{itemize}
    \item<1-> Erlang
      \begin{itemize}[type=1]
        \item<1-> Highlights
        \item<1-> Types and typing
      \end{itemize}
    \item<1-> Data representation
      \begin{itemize}[type=1]
        \item<1-> A way...
        \item<1-> Demos
      \end{itemize}
    \item<1-> Conclusion
      \begin{itemize}[type=1]
        \item<1-> Results
        \item<1-> Future work
      \end{itemize}
  \end{itemize}
}  
\end{slide}

%%%%%%%%%%%%%%%%%%%%%%%%%%%%%%%%%%%%%%%%%%%%%%%%%%%%%%%%%%%%%%%

\section{Introduction}                       %%% = slide 4

\begin{slide}{Context}                       %%% = slide 5
\vspace{-3mm}
  \begin{itemize}[type=1]
    \item<1-> ``Software Engineering'' (1968)
      \begin{itemize}[type=1]
        \item<1-> systematic, quantifiable, disciplined
        \item<1-> development, operation, and maintenance of software
      \end{itemize}
  \end{itemize}

\vspace{-3mm}
\onslide*{2-}{
  \begin{itemize}[type=1]
    \item<2-> ``Software Verification''
      \begin{itemize}[type=1]
        \item<2-> software engineering complex discipline
        \item<2-> software satisfies expected requirements
      \end{itemize}
  \end{itemize}
}  

\onslide*{3-}{
  Influencing factors:
}
\onslide*{3-}{
  \begin{itemize}[type=1]
    \item<3-> increase in computational power of general purpose
      computers
    \item<3-> increasing ability to analyse software
  \end{itemize}
}

\onslide*{4-}{
  Leading to a shift on the emphasis from abstract models of code,
  towards the implementation level code in software verification.
}
\end{slide}

%%-----------------------------------------------------------%%

\begin{slide}{Context (cont.)}               %%% = slide 5
Formal Software Verification
%\vspace{-3mm}
\begin{itemize}[type=1]
  \item<1-> Support for high-level languages
  \item<2-> Capacity to deal with complex algorithms
  \item<3-> Clear and simple specification
  \item<4-> Property expressiveness
  \item<5-> Easier (+/-) property verification
\end{itemize}

\vspace{7mm}
\onslide*{6-}{

Holy Graal: to be able to have most applications on a given system
formally verified and obeying to its specifications.
}

\end{slide}

%%-----------------------------------------------------------%%

\begin{slide}{Objective}               %%% = slide 5
Develop tool(s) for the verification of (a subset of) Erlang.
\begin{itemize}[type=1]
  \item<2-> Identify needs, work done and possible challenges
  \item<3-> Explore existing similar or intersecting tools (\texttt{Why})
  \item<4-> Define adequate abstract representations
  \item<5-> Test the representation with significant case studies
\end{itemize}

\vspace{7mm}
\onslide*{6-}{
  Take some conclusions in order to define and implement our own
  specialised mechanism of property analysis.
}
\end{slide}


%%-----------------------------------------------------------%%

%%%%%%%%%%%%%%%%%%%%%%%%%%%%%%%%%%%%%%%%%%%%%%%%%%%%%%%%%%%%%%%

\section{Erlang}                       %%% = slide 3

%%-----------------------------------------------------------%%

\begin{slide}{Highlights}                  %%% = slide 4

\onslide*{1-}{
  \begin{itemize}[type=1]
    \item Declarative
    \item Concurrent
    \item Soft real-time
    \item Hot code loading
    \item ...
  \end{itemize}
}
\vspace{5mm}
\onslide*{2-}{
  Language safety
  \begin{itemize}[type=1]
    \item No global variables -- fewer side-effects
    \item No direct memory access -- no pointer errors
    \item No malloc/free bugs
    \item Solid concurrency model -- reduces synchronisation problems, reduces the state space (simpler programs)
    \item Fault isolation -- memory-protected lightweight processes
    \item Built-in error recovery support -- more consistency
  \end{itemize}
}
\end{slide}

%%-----------------------------------------------------------%%

\begin{slide}{Types and typing}                  %%% = slide 4

\vspace{-4mm}
  Variables and values, or terms, are of one of the following types:
\begin{itemize}[type=1]
  \item integer
  \item list (of terms)
  \item tuples (of terms)
  \item pid
  \item atom
\end{itemize}

\vspace{2mm}
\onslide*{2-}{
Types are dynamically checked, leading to programs with less normal features
}
\onslide*{3-}{
  \begin{itemize}[type=1]
    \item odd types not going wrong due to control flow branching
    \item dependent types and polymorphism occur quite naturally
    \item wide use of union types
    \item ...
  \end{itemize}
}

\vspace{2mm}
\onslide*{4-}{
No one succeeded in building a type checker that works with all the
interesting Erlang programs...
}

\end{slide}

%%-----------------------------------------------------------%%

%%%%%%%%%%%%%%%%%%%%%%%%%%%%%%%%%%%%%%%%%%%%%%%%%%%%%%%%%%%%%%%

\section{Data representation}                       %%% = slide 3

\begin{slide}{A way...}                  %%% = slide 5\end{slide}
\vspace{3mm}
\onslide*{1-3}{
  If we want to statically analyse properties of values and
  computations, we need to find some strategy to deal with this kind
  of complexity!
}

\vspace{7mm}
\onslide*{2-3}{
  One solution is to represent everything with type {\it term()} and
  then use some type based techniques to ensure the desired
  properties.
}

\vspace{7mm}
\onslide*{3-3}{
  The representation of values of each type will be given by specific
  functional symbols and their behaviour by a small set of axioms.
}
\end{slide}

%%-----------------------------------------------------------%%

\begin{slide}{A way...(2)}                  %%% = slide 5\end{slide}
\vspace{-3mm}
\onslide*{1-}{
Step by step...
\begin{itemize}[type=1]
  \item<1->  give all variables from the original source code the
    \whyType{term} type
  \item<2->  encapsulate all values with the functional symbols we
    have defined in order to comply with the type
  \item<3->  annotate the code with it's desired specification
    \begin{itemize}[type=1]
      \item<4-> the requirements for the functions (pre-conditions)
      \item<4-> the properties ensured on the output/after execution
        (post-conditions)
    \end{itemize}
  \item<5->  arrange all to meet VCgen input specs
\end{itemize}
}

\vspace{7mm}
\onslide*{6-}{
  Now we have all we need, ``just'' run the verification condition
  generator and chose your proving media
}

%\vspace{7mm}
%\onslide*{3-}{
%}
\end{slide}

%%-----------------------------------------------------------%%

\begin{slide}{Demos...}                  %%% = slide 5\end{slide}
\vspace{3mm}
\onslide*{1-}{
Representation of values.
}

\vspace{2mm}
\onslide*{2-}{
\footnotesize
\whyReserved{type} \whyType{list}\\
\whyReserved{type} \whyType{tuple}\\

\vspace{2mm}
\whyComment{\scriptsize (* to build terms - logic operators from values to term *)}\\
\whyReserved{logic} \whyOther{langList}    \whyColon \whyType{list}  \whyArrow  \whyType{term}\\
\whyReserved{logic} \whyOther{langTuple}   \whyColon \whyType{tuple} \whyArrow  \whyType{term}\\
\whyReserved{logic} \whyOther{langInteger} \whyColon \whyType{int}   \whyArrow  \whyType{term}\\

\vspace{2mm}
\whyComment{\scriptsize (* to get terms value - logic operators from term to values *)}\\
\whyReserved{logic} \whyOther{getList}    \whyColon  \whyType{term}  \whyArrow  \whyType{list}\\
\whyReserved{logic} \whyOther{getTuple}   \whyColon  \whyType{term}  \whyArrow  \whyType{tuple}\\
\whyReserved{logic} \whyOther{getInteger} \whyColon  \whyType{term}  \whyArrow  \whyType{int}\\
}

\onslide*{3-}{
\vspace{4mm}
\whyReserved{axiom} \whyOther{getSetList} \whyColon\\
\quad  \whyReserved{forall} \whyOther{l} \whyColon \whyType{list}\whyDot
\qquad    \whyFunction{getList}\whyOP \whyFunction{langList}\whyOP \whyOther{l} \whyCP \whyCP \whyEq \whyOther{l}

}


\end{slide}

\begin{slide}{Demos...(2)}                  %%% = slide 5\end{slide}
\vspace{5mm}
\onslide*{1-}{
List definition...
}

\vspace{3mm}
\whyComment{\scriptsize (* list constructions *)}\\
\whyReserved{logic} \whyOther{nil} \whyColon \whyType{list}\\
\whyReserved{logic} \whyOther{cons} \whyColon \whyType{term}, \whyType{list}  \whyArrow  \whyType{list}\\

\vspace{3mm}
\whyReserved{axiom} \whyOther{langList\_ax} \whyColon\\
\quad  \whyReserved{forall} \whyOther{l} \whyColon \whyType{list}\whyDot \\
\qquad  \whyReserved{exists} \whyOther{rl} \whyColon \whyType{list}\whyDot \\
\quad\qquad    \whyReserved{exists} \whyOther{t} \whyColon \whyType{term}\whyDot \\
\qquad\qquad    \whyOther{l} \whyEq \whyOther{nil} \whyReserved{or}
\whyOther{l} \whyEq \whyFunction{cons}\whyOP\,\whyOther{t},\,\whyOther{rl}\whyCP



\end{slide}

%%-----------------------------------------------------------%%

\begin{slide}{List reverse}                  %%% = slide 5\end{slide}
\footnotesize
\vspace{8mm}

\whyReserved{let} \whyReserved{rec}  \whyVar{reverse\_aux}
  \whyOP\,\whyVar{L}\whyColon\whyType{term} \whyCP\,\,
  \whyOP\,\whyVar{Lprev}\whyColon\whyType{term} \whyCP\,\,
    \whyColon \whyType{term}\,\,\whyEq\\
\onslide*{1-1}{
\vspace{8mm}
}
\onslide*{2-}{
\vspace{2mm}
  \{ \whyFunction{isList}\whyOP\whyOther{L}\whyCP\whyEq\whyTrue
 \quad \whyAnd  \quad
 \whyFunction{isList}\whyOP\whyOther{Lprev}\whyCP\whyEq\whyTrue \}\\
\vspace{2,1mm}
}
    \whyIf \quad\whyFunction{isNil\whyOP getList L\whyCP}
        \whyEq\,\,\whyTrue\quad\whyThen\\
\quad      \whyOther{Lprev}\\
    \whyElse \quad
      \whyComment{(* cons(N,RL) *)}\\
\quad      \whyLet \quad \whyVar{H} \whyEq
          \whyFunction{head \whyOP getList L \whyCP} \whyIn\\
\quad      \whyLet \quad \whyVar{RL} \whyEq
          \whyFunction{tail \whyOP getList L \whyCP} \whyIn\\
\qquad      \whyOther{reverse\_aux \whyOP langList RL \whyCP\\
\qquad\qquad  \whyOP langList \whyOP cons H \whyOP getList Lprev
              \whyCP\whyCP\whyCP }\\
\onslide*{3-}{
\vspace{3mm}
  \{ \whyOther{langListLength\whyOP getList\whyOP result\whyCP\whyCP}
  \,\whyEq\,\\\quad\quad
  \whyOther{langListLength\whyOP getList\whyOP L\whyCP\whyCP\\
\qquad\quad
    $+$ langListLength\whyOP getList\whyOP Lprev\whyCP\whyCP}\}
}

% --=# NOT INCLUDED #=--
%    and (forall i:int.
%	   ( 1<=i<=erl_list_length(get_list(L)) )
%	        ->
%	   ( nth(i,get_list(L)) = nth(erl_list_length(get_list(L))+1-i,get_list(result)) )
	     
           
%        )
%    and (forall j:int.
%	   ( 1<=j<=erl_list_length(get_list(Lprev)) )
%	        ->
%	   ( nth(j,get_list(Lprev)) = nth(erl_list_length(get_list(L))+j,get_list(result)) )
	     
           
%        )
%  }
 

%\whyReserved{let} reverse =
%  fun (L:erl_term) ->
%  { is_list(L)=true }
%    (reverse_aux L (erlang_list nil))
%  { erl_list_length(get_list(L)) = erl_list_length(get_list(result)) 
%    and (forall i:int.
%	   1<=i<=erl_list_length(get_list(L))
%	     -> nth(i,get_list(L)) = nth(erl_list_length(get_list(L))+1-i,get_list(result))
%	)
%  }





\end{slide}

%%-----------------------------------------------------------%%

\section{Conclusion}

\begin{slide}{Results}                  %%% = slide 9
\vspace{3mm}
\onslide*{1-}{
It appears possible to model Erlang data for simple type checking.
}

\vspace{6mm}
\onslide*{2-}{
We are still able to enhance the details in order to reason about
intended properties.
}

\vspace{6mm}
\onslide*{3-}{
Impossible to reuse some tools used in verification of C/Java code.
}

\vspace{6mm}
\onslide*{4-}{
This have to be way more automatic...
}
\end{slide}

\begin{slide}{Future Work}                  %%% = slide 9
  Future work on this topic includes:
  \begin{itemize}[type=1]
    \item<1-> representation of other data types (tuples)
    \item<2-> verification of a bigger algorithm (AVL trees)
    \item<3-> interaction between terms of different type
    \item<4-> automatising of the process
%    \item<5-> 
%    \item<->
%    \item<9-> increasing ability to analyze software
  \end{itemize}

\onslide*{5-}{
  Other future objective consist on building a tool that:
  \begin{itemize}[type=1]
    \item<6-> support higher level constructs and features
    \item<7-> has some kind of support to concurrency
    \item<8-> has an interface closer to the the source language
    \item<9-> is partially automated
  \end{itemize}
}

\end{slide}

\section{The End}

\end{document}
