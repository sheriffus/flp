% -*- mode: latex; mode: folding -*-

\section{Case Studies}
\label{sec:case-studies}

\subsection{List Processing}
We started to write small predicates and specifications along with the
tool development.
%
These small examples turned out to be interesting to show some of the
potential of \plqc{}.
%
They helped to reveal discrepancies between intended and actual
behaviour of the predicates.
%
The tool is specially useful for people that are beginning logic
programming or have their logic programming polluted by other
programming styles, such as one of the authors.


\subsubsection{List Reverse}
We start by extending the first example given in
\refSec{sec:properties}.
%
In general people have a conception of what is reversing the order of
elements, which is to observe the $i^th$ element of the original list as
the $(1+n-i)^th$ element of the reversed one for a list of size $n$.
%
There is also the idea of how to code such a procedure, independently of
the programming language.
%
In fact, there are quite a few ways to implement it and we played with
some implementations of the procedure.


What we intend to check about our reverse procedures is the ability to
transform an input list into its reversed counterpart.
%
This implies that the property has an input for random testing.
%
For the input we must 
\subsection{AVL Trees}
\subsection{red-black Trees}



