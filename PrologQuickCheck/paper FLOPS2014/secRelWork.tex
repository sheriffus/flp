% -*- mode: latex; mode: folding -*-
\documentclass[runningheads,a4paper]{../../PaperStyles/llncs}

%% ODER: format ==         = "\mathrel{==}"
%% ODER: format /=         = "\neq "
%
%
\makeatletter
\@ifundefined{lhs2tex.lhs2tex.sty.read}%
  {\@namedef{lhs2tex.lhs2tex.sty.read}{}%
   \newcommand\SkipToFmtEnd{}%
   \newcommand\EndFmtInput{}%
   \long\def\SkipToFmtEnd#1\EndFmtInput{}%
  }\SkipToFmtEnd

\newcommand\ReadOnlyOnce[1]{\@ifundefined{#1}{\@namedef{#1}{}}\SkipToFmtEnd}
\usepackage{amstext}
\usepackage{amssymb}
\usepackage{stmaryrd}
\DeclareFontFamily{OT1}{cmtex}{}
\DeclareFontShape{OT1}{cmtex}{m}{n}
  {<5><6><7><8>cmtex8
   <9>cmtex9
   <10><10.95><12><14.4><17.28><20.74><24.88>cmtex10}{}
\DeclareFontShape{OT1}{cmtex}{m}{it}
  {<-> ssub * cmtt/m/it}{}
\newcommand{\texfamily}{\fontfamily{cmtex}\selectfont}
\DeclareFontShape{OT1}{cmtt}{bx}{n}
  {<5><6><7><8>cmtt8
   <9>cmbtt9
   <10><10.95><12><14.4><17.28><20.74><24.88>cmbtt10}{}
\DeclareFontShape{OT1}{cmtex}{bx}{n}
  {<-> ssub * cmtt/bx/n}{}
\newcommand{\tex}[1]{\text{\texfamily#1}}	% NEU

\newcommand{\Sp}{\hskip.33334em\relax}


\newcommand{\Conid}[1]{\mathit{#1}}
\newcommand{\Varid}[1]{\mathit{#1}}
\newcommand{\anonymous}{\kern0.06em \vbox{\hrule\@width.5em}}
\newcommand{\plus}{\mathbin{+\!\!\!+}}
\newcommand{\bind}{\mathbin{>\!\!\!>\mkern-6.7mu=}}
\newcommand{\rbind}{\mathbin{=\mkern-6.7mu<\!\!\!<}}% suggested by Neil Mitchell
\newcommand{\sequ}{\mathbin{>\!\!\!>}}
\renewcommand{\leq}{\leqslant}
\renewcommand{\geq}{\geqslant}
\usepackage{polytable}

%mathindent has to be defined
\@ifundefined{mathindent}%
  {\newdimen\mathindent\mathindent\leftmargini}%
  {}%

\def\resethooks{%
  \global\let\SaveRestoreHook\empty
  \global\let\ColumnHook\empty}
\newcommand*{\savecolumns}[1][default]%
  {\g@addto@macro\SaveRestoreHook{\savecolumns[#1]}}
\newcommand*{\restorecolumns}[1][default]%
  {\g@addto@macro\SaveRestoreHook{\restorecolumns[#1]}}
\newcommand*{\aligncolumn}[2]%
  {\g@addto@macro\ColumnHook{\column{#1}{#2}}}

\resethooks

\newcommand{\onelinecommentchars}{\quad-{}- }
\newcommand{\commentbeginchars}{\enskip\{-}
\newcommand{\commentendchars}{-\}\enskip}

\newcommand{\visiblecomments}{%
  \let\onelinecomment=\onelinecommentchars
  \let\commentbegin=\commentbeginchars
  \let\commentend=\commentendchars}

\newcommand{\invisiblecomments}{%
  \let\onelinecomment=\empty
  \let\commentbegin=\empty
  \let\commentend=\empty}

\visiblecomments

\newlength{\blanklineskip}
\setlength{\blanklineskip}{0.66084ex}

\newcommand{\hsindent}[1]{\quad}% default is fixed indentation
\let\hspre\empty
\let\hspost\empty
\newcommand{\NB}{\textbf{NB}}
\newcommand{\Todo}[1]{$\langle$\textbf{To do:}~#1$\rangle$}

\EndFmtInput
\makeatother
%

%
%
%
%
%
% This package provides two environments suitable to take the place
% of hscode, called "plainhscode" and "arrayhscode". 
%
% The plain environment surrounds each code block by vertical space,
% and it uses \abovedisplayskip and \belowdisplayskip to get spacing
% similar to formulas. Note that if these dimensions are changed,
% the spacing around displayed math formulas changes as well.
% All code is indented using \leftskip.
%
% Changed 19.08.2004 to reflect changes in colorcode. Should work with
% CodeGroup.sty.
%
\ReadOnlyOnce{polycode.fmt}%
\makeatletter

\newcommand{\hsnewpar}[1]%
  {{\parskip=0pt\parindent=0pt\par\vskip #1\noindent}}

% can be used, for instance, to redefine the code size, by setting the
% command to \small or something alike
\newcommand{\hscodestyle}{}

% The command \sethscode can be used to switch the code formatting
% behaviour by mapping the hscode environment in the subst directive
% to a new LaTeX environment.

\newcommand{\sethscode}[1]%
  {\expandafter\let\expandafter\hscode\csname #1\endcsname
   \expandafter\let\expandafter\endhscode\csname end#1\endcsname}

% "compatibility" mode restores the non-polycode.fmt layout.

\newenvironment{compathscode}%
  {\par\noindent
   \advance\leftskip\mathindent
   \hscodestyle
   \let\\=\@normalcr
   \let\hspre\(\let\hspost\)%
   \pboxed}%
  {\endpboxed\)%
   \par\noindent
   \ignorespacesafterend}

\newcommand{\compaths}{\sethscode{compathscode}}

% "plain" mode is the proposed default.
% It should now work with \centering.
% This required some changes. The old version
% is still available for reference as oldplainhscode.

\newenvironment{plainhscode}%
  {\hsnewpar\abovedisplayskip
   \advance\leftskip\mathindent
   \hscodestyle
   \let\hspre\(\let\hspost\)%
   \pboxed}%
  {\endpboxed%
   \hsnewpar\belowdisplayskip
   \ignorespacesafterend}

\newenvironment{oldplainhscode}%
  {\hsnewpar\abovedisplayskip
   \advance\leftskip\mathindent
   \hscodestyle
   \let\\=\@normalcr
   \(\pboxed}%
  {\endpboxed\)%
   \hsnewpar\belowdisplayskip
   \ignorespacesafterend}

% Here, we make plainhscode the default environment.

\newcommand{\plainhs}{\sethscode{plainhscode}}
\newcommand{\oldplainhs}{\sethscode{oldplainhscode}}
\plainhs

% The arrayhscode is like plain, but makes use of polytable's
% parray environment which disallows page breaks in code blocks.

\newenvironment{arrayhscode}%
  {\hsnewpar\abovedisplayskip
   \advance\leftskip\mathindent
   \hscodestyle
   \let\\=\@normalcr
   \(\parray}%
  {\endparray\)%
   \hsnewpar\belowdisplayskip
   \ignorespacesafterend}

\newcommand{\arrayhs}{\sethscode{arrayhscode}}

% The mathhscode environment also makes use of polytable's parray 
% environment. It is supposed to be used only inside math mode 
% (I used it to typeset the type rules in my thesis).

\newenvironment{mathhscode}%
  {\parray}{\endparray}

\newcommand{\mathhs}{\sethscode{mathhscode}}

% texths is similar to mathhs, but works in text mode.

\newenvironment{texthscode}%
  {\(\parray}{\endparray\)}

\newcommand{\texths}{\sethscode{texthscode}}

% The framed environment places code in a framed box.

\def\codeframewidth{\arrayrulewidth}
\RequirePackage{calc}

\newenvironment{framedhscode}%
  {\parskip=\abovedisplayskip\par\noindent
   \hscodestyle
   \arrayrulewidth=\codeframewidth
   \tabular{@{}|p{\linewidth-2\arraycolsep-2\arrayrulewidth-2pt}|@{}}%
   \hline\framedhslinecorrect\\{-1.5ex}%
   \let\endoflinesave=\\
   \let\\=\@normalcr
   \(\pboxed}%
  {\endpboxed\)%
   \framedhslinecorrect\endoflinesave{.5ex}\hline
   \endtabular
   \parskip=\belowdisplayskip\par\noindent
   \ignorespacesafterend}

\newcommand{\framedhslinecorrect}[2]%
  {#1[#2]}

\newcommand{\framedhs}{\sethscode{framedhscode}}

% The inlinehscode environment is an experimental environment
% that can be used to typeset displayed code inline.

\newenvironment{inlinehscode}%
  {\(\def\column##1##2{}%
   \let\>\undefined\let\<\undefined\let\\\undefined
   \newcommand\>[1][]{}\newcommand\<[1][]{}\newcommand\\[1][]{}%
   \def\fromto##1##2##3{##3}%
   \def\nextline{}}{\) }%

\newcommand{\inlinehs}{\sethscode{inlinehscode}}

% The joincode environment is a separate environment that
% can be used to surround and thereby connect multiple code
% blocks.

\newenvironment{joincode}%
  {\let\orighscode=\hscode
   \let\origendhscode=\endhscode
   \def\endhscode{\def\hscode{\endgroup\def\@currenvir{hscode}\\}\begingroup}
   %\let\SaveRestoreHook=\empty
   %\let\ColumnHook=\empty
   %\let\resethooks=\empty
   \orighscode\def\hscode{\endgroup\def\@currenvir{hscode}}}%
  {\origendhscode
   \global\let\hscode=\orighscode
   \global\let\endhscode=\origendhscode}%

\makeatother
\EndFmtInput
%
% to handle ∀ a . t without making the dot be composition
% the `doubleequals' macro is due to Jeremy Gibbons
\def\doubleequals{\mathrel{\unitlength 0.01em
  \begin{picture}(78,40)
    \put(7,34){\line(1,0){25}} \put(45,34){\line(1,0){25}}
    \put(7,14){\line(1,0){25}} \put(45,14){\line(1,0){25}}
  \end{picture}}}
% If you remove the %format == command the lhs2TeX default yields ≡, which can be a problem

\newcommand{\shortequals}{
  {\unitlength 0.01em
  \begin{picture}(39,40)
    \put(7,34){\line(1,0){25}}
    \put(7,14){\line(1,0){25}}
  \end{picture}}
}

\newcommand{\doubledotequals}{\ensuremath{\mathrel{
  \shortequals . \shortequals}}}

% %format ^ = " "


% -*- mode: latex; mode: folding -*-

\usepackage{amsmath}
\usepackage{listings}
\usepackage{url}

\lstset{language=Prolog}          % Set your language (you can change
                                % the language for each code-block
                                % optionally)
%style for prolog programs
\lstdefinestyle{yap}{
  language=Prolog,
  basicstyle=\fontsize{8}{9.6}\ttfamily,
  keywordstyle=\ttfamily,
  showstringspaces=false,
  otherkeywords={ <->, (,) },
}
\lstnewenvironment{yapcode}
  {\lstset{style=yap}\lstset{basicstyle=\fontsize{8}{9.6}\ttfamily}}
  {}

\newcommand{\yap}[1]{\lstinline[style=yap]{#1}}

\newcommand{\todo}[2][?]{\marginpar{\raggedright \tiny TODO: #2}}
\newcommand{\refSec}[1]{Sec.~\ref{#1}}
\newcommand{\refSecI}[1]{Section~\ref{#1}}
\newcommand{\refFig}[1]{Fig.~\ref{#1}}
\newcommand{\refFigI}[1]{Figure~\ref{#1}}
\newcommand{\refFigs}[1]{Figs~\ref{#1}}
\newcommand{\refTab}[1]{Tab.~\ref{#1}}
\newcommand{\refTabI}[1]{Table~\ref{#1}}

\newcommand{\Yap}[0]{{\sf Yap}}
\newcommand{\Prolog}[0]{{\sf Prolog}}
\newcommand{\Haskell}[0]{{\sf Haskell}}
\newcommand{\QuickCheck}[0]{{\sf QuickCheck}}
\newcommand{\plqc}[0]{{\sf PrologCheck}}

%% % ----------------------------------------------------


%% LNCS stuff
\newcommand{\keywords}[1]{\par\addvspace\baselineskip
\noindent\keywordname\enspace\ignorespaces#1}


\begin{document}
%%%%%%%%%%%
\newcommand{\papertitle}{{PrologCheck} -- property-based testing in {Prolog}}

\mainmatter  % start of an individual contribution

% first the title is needed
\title{\papertitle}

% a short form should be given in case it is too long for the running head
%% \titlerunning{Lecture Notes in Computer Science: Authors' Instructions}

\author{
  Cl\'audio Amaral$^{1,2}$ \and
  M\'ario Florido$^{1,2}$ \and
  V\'itor Santos Costa$^{1,3}$
}
\authorrunning{Cl\'audio Amaral \and M\'ario Florido \and V\'itor Santos Costa}

% the affiliations are given next; don't give your e-mail address
% unless you accept that it will be published
\institute{
DCC - Faculty of Sciences, University of Porto
\and LIACC - University of Porto \and CRACS - University of Porto
\\
\url{{coa,amf,vsc}@dcc.fc.up.pt}}


\maketitle
\begin{abstract}


We present \plqc{}, an automatic tool for property-based
testing of programs in the logic programming language \Prolog{} with
randomised test data generation.
%
The tool is inspired by the well known \QuickCheck, originally designed
for the functional programming language \Haskell{}. It includes features
that deal with specific characteristics of \Prolog{} such as its
relational nature (as opposed to \Haskell{}) and the absence of a strong
type discipline.


\plqc{} expressiveness stems from describing properties as
\Prolog{} goals.
%
It enables the definition of custom test data generators for
random testing tailored for the property to be tested.
%
Further, it allows the use of a predicate specification language that
supports types, modes and constraints on the number of successful
computations.
%
We evaluate our tool on a number of  examples and apply
it successfully to debug a \Prolog{} library for AVL search trees.

\end{abstract}



% -*- mode: latex; mode: folding -*-

\section{Introduction}
\label{sec:intro}
%{{{ introduction text


Software testing consists of executing a program on
a pre-selected set of inputs and inspecting whether the outputs
respect the expected results.
%
Each input tested is called a \emph{test case} and the set of inputs is
a \emph{test suite}.
%
Testing tries to find counter-examples and choosing the test cases to
this effect is often a difficult task.
%
The approach used can be  manual, with the tester designing test cases
one by one, or it can be automated to some extent, in this case
resorting to tools for case generation.
%
Ideally, the best approach would be automatic testing.



In a property-based framework test cases are automatically generated and run
from assertions about logical properties of the program.
%
Feedback is given to the user about their evaluation.
%
Property-based testing applications include black-box, white-box, unit,
integration and system
testing~\cite{Boberg-model-based-testing-erl}~\cite{Claessen-ranking-prog-blackbox,Claessen-find-race-cond-erl}.




Property-based testing naturally fits the logic programming paradigm. 
%
Assertions are first order formulas and thus easily encoded as program predicates.
%
Therefore, a property based approach to testing
is intuitive for the
logic programmer.



In this paper we introduce \plqc{}\footnote{The \plqc{} tool is available at \url{www.dcc.fc.up.pt/~coa/PrologCheck.html}.},
a property-based testing framework for \Prolog{}.
%
We further discuss two main contributions: a specification language for \Prolog{} predicates
and a translation procedure into testable properties.



In most programming languages interfaces to testing frameworks rely
on boolean functions, such as equality, to determine primitive properties.
%
\plqc{} states properties through a domain-specific language that
naturally supports domain quantification.
%
In this language primitive properties are \Prolog{} goals which can be
composed by \plqc{} property operators. 



\plqc{} testing consists on repetitively calling the goal for a large number of
test cases.
%
Input to such goals is based on \plqc{} value abstraction,
quantification over a domain represented by a randomised generator of
terms.
%
We implement randomised test case generation, which frees the user
from choosing input manually.
%
We include a number of predefined generators for relevant sets of
terms, such as integers, and combinators to help define new
generators.
%
Thus other generation techniques
\cite{Duregard-2012-feat}~\cite{naylor2007logic}~\cite{runciman2008smallcheck} can be
implemented  to complement the power of built-in generators.



We also define a language of testable predicate specifications including
types, modes and multiplicity, which the tester can use to encode
interesting properties of the predicate under test.
%
By specifying some aspects of a predicate in a proper specification
language it is possible to generate a \plqc{} property and check it.
%
This allows us to  use \plqc{} and its predicate specification to test a
number of non-trivial programs.


%}}} introduction


%{{{ paper outline
The rest of this paper is organised as follows.
%
We proceed with the presentation of motivating examples in
\refSec{sec:motiv-ex}. 
%
\refSecI{sec:rel-work} encloses the presentation of related work.
%
In \refSec{sec:properties} we introduce property definitions and their
testing in \plqc{} and in \refSec{sec:generators} we discuss details
about test case generation.
%
\refSecI{sec:pred-spec} describes the predicate specification language
and how to test the specifications.
%
A case study of AVL trees is presented \refSec{sec:case-studies}. 
%
We finalise with the conclusions in \refSec{sec:conclusion}
%}}} outline


% -*- mode: latex; mode: folding -*-
\documentclass[runningheads,a4paper]{../../PaperStyles/llncs}

%% ODER: format ==         = "\mathrel{==}"
%% ODER: format /=         = "\neq "
%
%
\makeatletter
\\text{\tt ifundefined\char123{}lhs2tex\char46{}lhs2tex\char46{}sty\char46{}read\char125{}\char37{}~~~\char123{}\char92{}}namedef{lhs2tex.lhs2tex.sty.read}{}%
   \newcommand\SkipToFmtEnd{}%
   \newcommand\EndFmtInput{}%
   \long\def\SkipToFmtEnd#1\EndFmtInput{}%
  }\SkipToFmtEnd

\newcommand\ReadOnlyOnce[1]{\\text{\tt ifundefined\char123{}\char35{}1\char125{}\char123{}\char92{}}namedef{#1}{}}\SkipToFmtEnd}
\usepackage{amstext}
\usepackage{amssymb}
\usepackage{stmaryrd}
\DeclareFontFamily{OT1}{cmtex}{}
\DeclareFontShape{OT1}{cmtex}{m}{n}
  {<5><6><7><8>cmtex8
   <9>cmtex9
   <10><10.95><12><14.4><17.28><20.74><24.88>cmtex10}{}
\DeclareFontShape{OT1}{cmtex}{m}{it}
  {<-> ssub * cmtt/m/it}{}
\newcommand{\texfamily}{\fontfamily{cmtex}\selectfont}
\DeclareFontShape{OT1}{cmtt}{bx}{n}
  {<5><6><7><8>cmtt8
   <9>cmbtt9
   <10><10.95><12><14.4><17.28><20.74><24.88>cmbtt10}{}
\DeclareFontShape{OT1}{cmtex}{bx}{n}
  {<-> ssub * cmtt/bx/n}{}
\newcommand{\tex}[1]{\text{\texfamily#1}}	% NEU

\newcommand{\Sp}{\hskip.33334em\relax}


\newcommand{\Conid}[1]{\mathit{#1}}
\newcommand{\Varid}[1]{\mathit{#1}}
\newcommand{\anonymous}{\kern0.06em \vbox{\hrule\\text{\tt width\char46{}5em\char125{}\char125{}~\char92{}newcommand\char123{}\char92{}plus\char125{}\char123{}\char92{}mathbin\char123{}\char43{}\char92{}\char33{}\char92{}\char33{}\char92{}\char33{}\char43{}\char125{}\char125{}~\char92{}newcommand\char123{}\char92{}bind\char125{}\char123{}\char92{}mathbin\char123{}\char62{}\char92{}\char33{}\char92{}\char33{}\char92{}\char33{}\char62{}\char92{}mkern\char45{}6\char46{}7mu\char61{}\char125{}\char125{}~\char92{}newcommand\char123{}\char92{}rbind\char125{}\char123{}\char92{}mathbin\char123{}\char61{}\char92{}mkern\char45{}6\char46{}7mu\char60{}\char92{}\char33{}\char92{}\char33{}\char92{}\char33{}\char60{}\char125{}\char125{}\char37{}~suggested~by~Neil~Mitchell~\char92{}newcommand\char123{}\char92{}sequ\char125{}\char123{}\char92{}mathbin\char123{}\char62{}\char92{}\char33{}\char92{}\char33{}\char92{}\char33{}\char62{}\char125{}\char125{}~\char92{}renewcommand\char123{}\char92{}leq\char125{}\char123{}\char92{}leqslant\char125{}~\char92{}renewcommand\char123{}\char92{}geq\char125{}\char123{}\char92{}geqslant\char125{}~\char92{}usepackage\char123{}polytable\char125{}~~\char37{}mathindent~has~to~be~defined~\char92{}}ifundefined{mathindent}%
  {\newdimen\mathindent\mathindent\leftmargini}%
  {}%

\def\resethooks{%
  \global\let\SaveRestoreHook\empty
  \global\let\ColumnHook\empty}
\newcommand*{\savecolumns}[1][default]%
  {\g\text{\tt addto}macro\SaveRestoreHook{\savecolumns[#1]}}
\newcommand*{\restorecolumns}[1][default]%
  {\g\text{\tt addto}macro\SaveRestoreHook{\restorecolumns[#1]}}
\newcommand*{\aligncolumn}[2]%
  {\g\text{\tt addto}macro\ColumnHook{\column{#1}{#2}}}

\resethooks

\newcommand{\onelinecommentchars}{\quad-{}- }
\newcommand{\commentbeginchars}{\enskip\{-}
\newcommand{\commentendchars}{-\}\enskip}

\newcommand{\visiblecomments}{%
  \let\onelinecomment=\onelinecommentchars
  \let\commentbegin=\commentbeginchars
  \let\commentend=\commentendchars}

\newcommand{\invisiblecomments}{%
  \let\onelinecomment=\empty
  \let\commentbegin=\empty
  \let\commentend=\empty}

\visiblecomments

\newlength{\blanklineskip}
\setlength{\blanklineskip}{0.66084ex}

\newcommand{\hsindent}[1]{\quad}% default is fixed indentation
\let\hspre\empty
\let\hspost\empty
\newcommand{\NB}{\textbf{NB}}
\newcommand{\Todo}[1]{$\langle$\textbf{To do:}~#1$\rangle$}

\EndFmtInput
\makeatother
%

%
%
%
%
%
% This package provides two environments suitable to take the place
% of hscode, called "plainhscode" and "arrayhscode". 
%
% The plain environment surrounds each code block by vertical space,
% and it uses \abovedisplayskip and \belowdisplayskip to get spacing
% similar to formulas. Note that if these dimensions are changed,
% the spacing around displayed math formulas changes as well.
% All code is indented using \leftskip.
%
% Changed 19.08.2004 to reflect changes in colorcode. Should work with
% CodeGroup.sty.
%
\ReadOnlyOnce{polycode.fmt}%
\makeatletter

\newcommand{\hsnewpar}[1]%
  {{\parskip=0pt\parindent=0pt\par\vskip #1\noindent}}

% can be used, for instance, to redefine the code size, by setting the
% command to \small or something alike
\newcommand{\hscodestyle}{}

% The command \sethscode can be used to switch the code formatting
% behaviour by mapping the hscode environment in the subst directive
% to a new LaTeX environment.

\newcommand{\sethscode}[1]%
  {\expandafter\let\expandafter\hscode\csname #1\endcsname
   \expandafter\let\expandafter\endhscode\csname end#1\endcsname}

% "compatibility" mode restores the non-polycode.fmt layout.

\newenvironment{compathscode}%
  {\par\noindent
   \advance\leftskip\mathindent
   \hscodestyle
   \let\\=\\text{\tt normalcr~~~~\char92{}let\char92{}hspre\char92{}\char40{}\char92{}let\char92{}hspost\char92{}\char41{}\char37{}~~~~\char92{}pboxed\char125{}\char37{}~~~\char123{}\char92{}endpboxed\char92{}\char41{}\char37{}~~~~\char92{}par\char92{}noindent~~~~\char92{}ignorespacesafterend\char125{}~~\char92{}newcommand\char123{}\char92{}compaths\char125{}\char123{}\char92{}sethscode\char123{}compathscode\char125{}\char125{}~~\char37{}~\char34{}plain\char34{}~mode~is~the~proposed~default\char46{}~\char37{}~It~should~now~work~with~\char92{}centering\char46{}~\char37{}~This~required~some~changes\char46{}~The~old~version~\char37{}~is~still~available~for~reference~as~oldplainhscode\char46{}~~\char92{}newenvironment\char123{}plainhscode\char125{}\char37{}~~~\char123{}\char92{}hsnewpar\char92{}abovedisplayskip~~~~\char92{}advance\char92{}leftskip\char92{}mathindent~~~~\char92{}hscodestyle~~~~\char92{}let\char92{}hspre\char92{}\char40{}\char92{}let\char92{}hspost\char92{}\char41{}\char37{}~~~~\char92{}pboxed\char125{}\char37{}~~~\char123{}\char92{}endpboxed\char37{}~~~~\char92{}hsnewpar\char92{}belowdisplayskip~~~~\char92{}ignorespacesafterend\char125{}~~\char92{}newenvironment\char123{}oldplainhscode\char125{}\char37{}~~~\char123{}\char92{}hsnewpar\char92{}abovedisplayskip~~~~\char92{}advance\char92{}leftskip\char92{}mathindent~~~~\char92{}hscodestyle~~~~\char92{}let\char92{}\char92{}\char61{}\char92{}}normalcr
   \(\pboxed}%
  {\endpboxed\)%
   \hsnewpar\belowdisplayskip
   \ignorespacesafterend}

% Here, we make plainhscode the default environment.

\newcommand{\plainhs}{\sethscode{plainhscode}}
\newcommand{\oldplainhs}{\sethscode{oldplainhscode}}
\plainhs

% The arrayhscode is like plain, but makes use of polytable's
% parray environment which disallows page breaks in code blocks.

\newenvironment{arrayhscode}%
  {\hsnewpar\abovedisplayskip
   \advance\leftskip\mathindent
   \hscodestyle
   \let\\=\\text{\tt normalcr~~~~\char92{}\char40{}\char92{}parray\char125{}\char37{}~~~\char123{}\char92{}endparray\char92{}\char41{}\char37{}~~~~\char92{}hsnewpar\char92{}belowdisplayskip~~~~\char92{}ignorespacesafterend\char125{}~~\char92{}newcommand\char123{}\char92{}arrayhs\char125{}\char123{}\char92{}sethscode\char123{}arrayhscode\char125{}\char125{}~~\char37{}~The~mathhscode~environment~also~makes~use~of~polytable\char39{}s~parray~~\char37{}~environment\char46{}~It~is~supposed~to~be~used~only~inside~math~mode~~\char37{}~\char40{}I~used~it~to~typeset~the~type~rules~in~my~thesis\char41{}\char46{}~~\char92{}newenvironment\char123{}mathhscode\char125{}\char37{}~~~\char123{}\char92{}parray\char125{}\char123{}\char92{}endparray\char125{}~~\char92{}newcommand\char123{}\char92{}mathhs\char125{}\char123{}\char92{}sethscode\char123{}mathhscode\char125{}\char125{}~~\char37{}~texths~is~similar~to~mathhs\char44{}~but~works~in~text~mode\char46{}~~\char92{}newenvironment\char123{}texthscode\char125{}\char37{}~~~\char123{}\char92{}\char40{}\char92{}parray\char125{}\char123{}\char92{}endparray\char92{}\char41{}\char125{}~~\char92{}newcommand\char123{}\char92{}texths\char125{}\char123{}\char92{}sethscode\char123{}texthscode\char125{}\char125{}~~\char37{}~The~framed~environment~places~code~in~a~framed~box\char46{}~~\char92{}def\char92{}codeframewidth\char123{}\char92{}arrayrulewidth\char125{}~\char92{}RequirePackage\char123{}calc\char125{}~~\char92{}newenvironment\char123{}framedhscode\char125{}\char37{}~~~\char123{}\char92{}parskip\char61{}\char92{}abovedisplayskip\char92{}par\char92{}noindent~~~~\char92{}hscodestyle~~~~\char92{}arrayrulewidth\char61{}\char92{}codeframewidth~~~~\char92{}tabular\char123{}}{}\ensuremath{\Varid{p}\;\{\mskip1.5mu \lambda \Varid{linewidth}\mathbin{-}\mathrm{2}\lambda \Varid{arraycolsep}\mathbin{-}\mathrm{2}\lambda \Varid{arrayrulewidth}\mathbin{-}\mathrm{2}\;\Varid{pt}\mskip1.5mu\}}\text{\tt \char123{}\char125{}\char125{}\char37{}~~~~\char92{}hline\char92{}framedhslinecorrect\char92{}\char92{}\char123{}\char45{}1\char46{}5ex\char125{}\char37{}~~~~\char92{}let\char92{}endoflinesave\char61{}\char92{}\char92{}~~~~\char92{}let\char92{}\char92{}\char61{}\char92{}}normalcr
   \(\pboxed}%
  {\endpboxed\)%
   \framedhslinecorrect\endoflinesave{.5ex}\hline
   \endtabular
   \parskip=\belowdisplayskip\par\noindent
   \ignorespacesafterend}

\newcommand{\framedhslinecorrect}[2]%
  {#1[#2]}

\newcommand{\framedhs}{\sethscode{framedhscode}}

% The inlinehscode environment is an experimental environment
% that can be used to typeset displayed code inline.

\newenvironment{inlinehscode}%
  {\(\def\column##1##2{}%
   \let\>\undefined\let\<\undefined\let\\\undefined
   \newcommand\>[1][]{}\newcommand\<[1][]{}\newcommand\\[1][]{}%
   \def\fromto##1##2##3{##3}%
   \def\nextline{}}{\) }%

\newcommand{\inlinehs}{\sethscode{inlinehscode}}

% The joincode environment is a separate environment that
% can be used to surround and thereby connect multiple code
% blocks.

\newenvironment{joincode}%
  {\let\orighscode=\hscode
   \let\origendhscode=\endhscode
   \def\endhscode{\def\hscode{\endgroup\def\\text{\tt currenvir\char123{}hscode\char125{}\char92{}\char92{}\char125{}\char92{}begingroup\char125{}~~~~\char37{}\char92{}let\char92{}SaveRestoreHook\char61{}\char92{}empty~~~~\char37{}\char92{}let\char92{}ColumnHook\char61{}\char92{}empty~~~~\char37{}\char92{}let\char92{}resethooks\char61{}\char92{}empty~~~~\char92{}orighscode\char92{}def\char92{}hscode\char123{}\char92{}endgroup\char92{}def\char92{}}currenvir{hscode}}}%
  {\origendhscode
   \global\let\hscode=\orighscode
   \global\let\endhscode=\origendhscode}%

\makeatother
\EndFmtInput
%
% to handle ∀ a . t without making the dot be composition
% the `doubleequals' macro is due to Jeremy Gibbons
\def\doubleequals{\mathrel{\unitlength 0.01em
  \begin{picture}(78,40)
    \put(7,34){\line(1,0){25}} \put(45,34){\line(1,0){25}}
    \put(7,14){\line(1,0){25}} \put(45,14){\line(1,0){25}}
  \end{picture}}}
% If you remove the %format == command the lhs2TeX default yields ≡, which can be a problem

\newcommand{\shortequals}{
  {\unitlength 0.01em
  \begin{picture}(39,40)
    \put(7,34){\line(1,0){25}}
    \put(7,14){\line(1,0){25}}
  \end{picture}}
}

\newcommand{\doubledotequals}{\ensuremath{\mathrel{
  \shortequals . \shortequals}}}

% %format ^ = " "


% -*- mode: latex; mode: folding -*-

\usepackage{amsmath}
\usepackage{listings}
\usepackage{url}

\lstset{language=Prolog}          % Set your language (you can change
                                % the language for each code-block
                                % optionally)
%style for prolog programs
\lstdefinestyle{yap}{
  language=Prolog,
  basicstyle=\fontsize{8}{9.6}\ttfamily,
  keywordstyle=\ttfamily,
  showstringspaces=false,
  otherkeywords={ <->, (,) },
}
\lstnewenvironment{yapcode}
  {\lstset{style=yap}\lstset{basicstyle=\fontsize{8}{9.6}\ttfamily}}
  {}

\newcommand{\yap}[1]{\lstinline[style=yap]{#1}}

\newcommand{\todo}[2][?]{\marginpar{\raggedright \tiny TODO: #2}}
\newcommand{\refSec}[1]{Sec.~\ref{#1}}
\newcommand{\refSecI}[1]{Section~\ref{#1}}
\newcommand{\refFig}[1]{Fig.~\ref{#1}}
\newcommand{\refFigI}[1]{Figure~\ref{#1}}
\newcommand{\refFigs}[1]{Figs~\ref{#1}}
\newcommand{\refTab}[1]{Tab.~\ref{#1}}
\newcommand{\refTabI}[1]{Table~\ref{#1}}

\newcommand{\Yap}[0]{{\sf Yap}}
\newcommand{\Prolog}[0]{{\sf Prolog}}
\newcommand{\Haskell}[0]{{\sf Haskell}}
\newcommand{\QuickCheck}[0]{{\sf QuickCheck}}
\newcommand{\plqc}[0]{{\sf PrologCheck}}

%% % ----------------------------------------------------


%% LNCS stuff
\newcommand{\keywords}[1]{\par\addvspace\baselineskip
\noindent\keywordname\enspace\ignorespaces#1}


\begin{document}
%%%%%%%%%%%
\newcommand{\papertitle}{{PrologCheck} -- property-based testing in {Prolog}}

\mainmatter  % start of an individual contribution

% first the title is needed
\title{\papertitle}

% a short form should be given in case it is too long for the running head
%% \titlerunning{Lecture Notes in Computer Science: Authors' Instructions}

\author{
  Cl\'audio Amaral$^{1,2}$ \and
  M\'ario Florido$^{1,2}$ \and
  V\'itor Santos Costa$^{1,3}$
}
\authorrunning{Cl\'audio Amaral \and M\'ario Florido \and V\'itor Santos Costa}

% the affiliations are given next; don't give your e-mail address
% unless you accept that it will be published
\institute{
DCC - Faculty of Sciences, University of Porto
\and LIACC - University of Porto \and CRACS - University of Porto
\\
\url{{coa,amf,vsc}