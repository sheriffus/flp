% --------------------------------------------------- Slide --
\begin{frame}[plain]
  \titlepage
  \begin{center} {
      \pgfuseimage{university-logo} \\ 
      
      \vspace{0.1cm}
      %\small Discussion leader: name, name University, city.
    }
  \end{center}
\end{frame}
% --------------------------------------------------- Section --
%\section[Contents]{}
% --------------------------------------------------- Slide ----
\begin{frame}
	\frametitle{Outline}
	\tableofcontents[%
%		currentsection, % causes all sections but the current to be shown in a semi-transparent way.
% 		currentsubsection, % causes all subsections but the current subsection in the current section to ...
% 		hideallsubsections, % causes all subsections to be hidden.
%		hideothersubsections, % causes the subsections of sections other than the current one to be hidden.
% 		part=, % part number causes the table of contents of part part number to be shown
%		pausesections, % causes a \pause command to be issued before each section. This is useful if you
% 		pausesubsections, %  causes a \pause command to be issued before each subsection.
% 		sections={ overlay specification },
	]
\end{frame}
% --------------------------------------------------------------
% --------------------------------------------------- Section --
\section{Intro}
% --------------------------------------------------- SubSec ---
\subsection{Automated stuff}
% --------------------------------------------------- Slide ----
\frame{
  \frametitle{}

  \note{definition for those of us mere mortals}
  \begin{definition}
    Computer programs that allow
    \only<3->{\alert<3>{(dis)}}proving
    \only<1-3>{of}\only<4->{{that}} mathematical
    \only<1>{theorems.}\only<2->{\alert<2>{conjectures}\only<2>{.}}
    \only<4->{\alert<4>{ are logical consequences of a set of
        statements (the axioms and hypotheses)}.}
  \end{definition}

  \vspace{2cm}
  \uncover<5->{
    What are most people looking for?
  }

}
% --------------------------------------------------- Slide ----
\frame{
  \frametitle{...}

  \note{  }
  \begin{definition}[]
    ooooooo ooooooooooooooo ooooooo oooooooooooo ooooooooooooooo oo 
    ooooooo ooooooooo ooooo oooo \cite{}.
  \end{definition}

  \note{To understand  first
    understand what  is.}

  \uncover<2->{
    \begin{definition}[Reasoning \cite{}]
      Process of drawing conclusions from facts
      \begin{itemize}
      \item<3-> conclusions must follow inevitably from the facts
\note{in other words}
      \item<4-> not concerned with some conclusion that has a good
        chance of being true when the facts are true
      \item<5-> refers to logical reasoning, not of common-sense or
        probabilistic reasoning
      \end{itemize}
       
    \end{definition}
  }

}
% --------------------------------------------------- Slide ----
\frame{
  \frametitle{}

  \uncover<1->{
  }
  \note{}

  \vspace{1cm}
  \uncover<2->{
  }

  \vspace{1cm}
  \uncover<3->{
  }

  \note{}
}

% --------------------------------------------------------------
secIntro.tex
% --------------------------------------------------------------
\section{}

   \begin{frame}
       \frametitle{To more juicy stuff...}
       \tableofcontents[currentsection]
   \end{frame}

%\subsection[abrev]{SubSecTitle}
\subsection[Approaches]{Some approaches to}

\frame{
  \frametitle{}

\begin{center}
  \begin{tabular}{l p{9cm}}
  1950s: & Turing, von Neuman: Hand proofs of program
  correctness
  \note{mathematical proofs, not much formal}
\\
  1966/67: & Floyd introduces assertional reasoning on
  flowcharts for proving partial and total correctness.
\\
  1969: & Hoare introduces axiomatic semantics for
  programming constructs.
  \note{basically equivalent to Floyd's thing, but based on a
    programming language and not a flowchart language}
  \\
  & King writes a dissertation on automatic program proving
  through verification condition generation.
  \note{
    based on Floyd's theory and with a theorem prover
    theorems were implications between formulas in disjunctive normal
    form; proof by contradiction - split into smaller formulas and
    check 'false' with a linear solver based on [Kuhn 56] method
  }
\\
  1970s: & Predicate transformer semantics, LCF/ML,
  Boyer-Moore prover, abstract interpretation,
  temporal logic, combination decision procedures.
\\
  1980s: & Model checking, hardware verification, HOL/Coq/Isabelle,
  UNITY, TLA, I/O automata, Z specification language, OBJ3, KIDS.
\\
  1990s: & Symbolic model checking, timed/hybrid model checking,
  predicate abstraction, bounded model checking, B Method, proof
  carrying code, typed assembly language. 
\\
\end{tabular}
\end{center}
}






\subsection{Why tool flash demonstration}


\frame{
\frametitle{Demo}
Any questions so far?

\pause
\vspace{2mm}
Before we finish, and breaking the line of thought of this talk (if
there was one)...
}

% --------------------------------------------------------------


\section[Concl]{Conclusions}

\frame{
  \frametitle{Conclusions}

  ooooooooooo ooooooooo ooo oooo oooooooooo  oo  o  ooooooo ooooooo oo
  ooo oooooooooooo oo ooooooooooooo oooo o.

  \vspace{6mm}
  \only<1-3>{\uncover<2-3>{
    oo ooooooo, oooooooooooo oo ooooooooo  ooooooooo oooooooo ooo 
    ooooood, ooo oo oo oo oo oo oooooo  oo ooo oooooooooooo oo
    ooooooooooo. \only<3>{\cite{ooooooooooooo:ooooooooooooooo}}
  }}
  \only<4->{
    oooooooooo oo ooo (ooo oooo oooooos) oooooooooos, ooooooood oooooom
    oooooog oos oooo oooooooooooy ooooood oo oooooooood ooos. oo ooo ooooo
    ooo ooo ooo oo, ooo oooooooo ooo.
  }

  \vspace{6mm}
  \uncover<5->{
    ooooooooo oooo ooooooo oooooo ooooooo ooo ooooo oo ooooooo ooo oooooo
    ooo ooooooo oooooooo oo o \only<5>{ooo \alert{oooooooooo
        oooooooo}}\only<6->{Cooooooo ooooooo}.
  }
  \note{hope it has given some flavour of this vital and fascinating
    research field.} 

}

% --------------------------------------------------------------

\frame{
\frametitle{}
}




% --------------------------------------------------- PART -----
%\part{Tutorial}  % separate TOC
%\frame{\partpage}
% --------------------------------------------------- Section --
%\section{SecTitle}
% --------------------------------------------------- SubSec ---
%\subsection{SubSecTitle}
% --------------------------------------------------------------
% --------------------------------------------------- Slide ----
% % --------------------------------------------------- Slide ----
% \begin{frame}
%
% \end{frame}

% --------------------------------------------------------------
\begin{thebibliography}{10}
\bibitem{john_harrison:AR_short_survey}[Harrison, 2007]
  John Harrison.
  \newblock A short survey of automated reasoning.
  \newblock {\footnotesize In \emph{Proceedings of the 2nd
      international conference on Algebraic biology} (AB'07), Hirokazu
    Anai, Katsuhisa Horimoto, and Temur Kutsia
    (Eds.). Springer-Verlag, Berlin, Heidelberg, 334-349.
  }
\bibitem{h_wang:to_mech_math_1960}[Wang, 1960]
  Hao Wang.
  \newblock Toward mechanical mathematics.
  \newblock {\footnotesize \emph{IBM Journal of research and development}, 
    4:2–22, 1960.
  }
\bibitem{r_castello_r_mili:thm_provers_survey1998}[Castelló and Mili, 1998]
  R. Castelló and R. Mili.
  \newblock Theorem Provers Survey.
\bibitem{w_w_bledsoe_l_j_henschen:what_is_atp1985}[Bledsoe and Henschen, 1985]
  Woodrow W. Bledsoe and L. J. Henschen.
  \newblock What is automated theorem proving?
  \newblock {\footnotesize \emph{Journal of Automated Reasoning},
    1(1):23-28, 1985.
  }
\bibitem{l_wos:what_is_ar1985}[Wos, 1985]  % 5
  Larry Wos.
  \newblock What is automated reasoning?
  \newblock {\footnotesize \emph{Journal of Automated Reasoning},
    1(1):6-8, 1985.
  }
\bibitem{w_mccune_r_padmanhabhan:aut_ded_in_eq_logic1996}[McCune and Padmanabhan, 1996]
  W. McCune and R. Padmanabhan.
  \newblock Automated Deduction in Equational Logic and Cubic Curves.
  \newblock {\footnotesize \emph{Lecture Notes in Computer Science},
    volume 1095, Springer-Verlag, 1996.
  }
\bibitem{andreas_folkler_msc:atp-res_vs_tab2002}[Folkler, 2002]
  Andreas L. E. Folkler.
  \newblock \emph{Automated Theorem Proving: Resolution vs Tableaux}.
  \newblock {\footnotesize \emph{(MSc thesis)}, Blekinge Institute of
    Technology.
  }
\bibitem{h_gel_j_r_han_d_w_lov:empir_expl_geom_thm_machine1960}[Gelernter,
  Hansen and Loveland , 1960]
  H. Gelernter, J. R. Hansen, and D. W. Loveland.
  \newblock Empirical explorations of the geometry theorem machine.
  \newblock {\footnotesize In \emph{Papers presented at the May 3-5,
      1960, western joint IRE-AIEE-ACM computer conference
      (IRE-AIEE-ACM '60 (Western)).} ACM, New York, NY, USA,
    143-149.
  }
\bibitem{r_jhala_r_majumdar:sw_mc2009}[Jhala and Majumdar, 2009]
  Ranjit Jhala and Rupak Majumdar.
  \newblock Software model checking.
  \newblock {\footnotesize \emph{ACM Comput. Surv. 41}, 4, Article 21
    (October 2009), 54 pages.
  }
  

%% \bibitem{ref_name}[authLastName(s), year]
%%   Complete(r) name(s).
%%   \newblock Title,
%%   \newblock {\footnotesize \emph{Venue}, more info.
%%   }
\end{thebibliography}

