% -*- mode: latex; mode: folding -*-
%% \pagebreak

\section{Conclusion}
\label{sec:conclusion}

%% %
%% (Find place for)
%% The tool is specially useful for people that are beginning logic
%% programming or that have their logic programming polluted by other
%% programming styles (arguably,  such as one of the authors).


Here we present \plqc{}, an automatic tool for specification based
testing of \Prolog{} programs.
%
Compared to similar tools for functional languages, we deal with testing
of nondeterministic programs in a logic programming language.
%
We provide a language to write properties, with convenient features,
such as quantifiers, conditionals, directionality and multiplicity.
%
\plqc{} also includes the notion of random test-data generation.
%
%% There are differences between \plqc{} generators and the generators
%% in a strongly typed version such as \Haskell\ \QuickCheck.
%% %
%% In a strongly typed language, generators pick values with a specific
%% type.
%% %
%% In \plqc{} generators are special predicates that randomly choose an
%% element from a set of values.
%% %
%% The generators define the set by the elements they
%% generate, regardless of their probability distribution.
%% %
%% Thus \plqc{} generators may be heterogenous, in the sense of
%% defining collections of terms with different types.


We show that specification based testing works extremely well for
\Prolog.
%
The relational nature of the language allows to specify local properties
quite well since all the dependencies between arguments are explicit in
predicate definitions.


Finally note that our tool uses \Prolog{} also to write properties, which,
besides its use in the tool for test specification, increases the
understanding of the program itself, without an extra learning for
\Prolog{} programmers.
