% -*- mode: latex; mode: folding -*-

\section{Evaluating laws}
%% \section{Equality checking}
\label{sec:equalities}

The satisfiability of the (state) monad laws is checked by the
evaluation of a binary relation on (state) monadic expressions.
%
This relation is denoted in \refFig{fig:laws} as (|===|) and represents
equational equivalence that (state) monadic expressions should obey.


It is not clear from the laws what the exact semantics of the (|===|)
relation is.
%
Expressions have to be \emph{equal} in some sense.
%
For complex structures such as the standard implementations of the
state monad there is more than one way to check equivalence of
expressions.


For common types (lifted sets, products and sums) the usual recursive
notion of equality is the operation that better represents equivalence
of values of such types.
%
\[
\forall x \in t. \quad x =_E x
\]
But (state) monadic values are more complex and need to be carefully
addressed.


\paragraph{Exact equality.} % (|(==)|)}
Exact equality of monadic values examines the exact domain elements that
the (monadic) values refer to.
%
This equality check is aware of the implementation of the state monad
type and compares the underlying structure for equality.
%
Values of common types are compared as previously mentioned.
%
For the lifted function space we compare the objects that represent
them in the respective domain.
%
\todo{worth doing?\\
  In refFig\{UnitBoolDom\} we can see an example of a domain of the
  function space.
  %
  It is the domain of functions from \emph{unit} (one-element set) to
  \emph{bool} (two-element set).
}
%
Any function will either be the |bottom| function or result in one of
the possible maps domain/co-domain, independently of differences in
the implementation.
%
Intuitively, two function values are equal if they are either both
|bottom| or if they map all elements of the domain to equal images.
%
\begin{align}
\forall & f, g \in (a \rightarrow b). \notag\\
%  & 
\quad & f =_E g \iff \lnot (f = \bot \land g = \bot)
%\notag\\
    \Rightarrow (\forall x \in a. \ f x =_E g x)
\notag
\end{align}


\paragraph{Run equality.} % (|. runS  ==  . runS|)}
Another way to compare complex expressions is to compute the
result of the expressions. 
%
We can use the state monadic interpretations available and compare the
results of such interpretations.
%
The state monads in \refSec{sec:stm} all present the |runS|
interpretation function.
%
This function \emph{runs} the state monadic computation from some
initial state to a pair of the final state and the result value of the
monad.
%
We proceed to check equality of the result values to determine the
equality of the original monadic expressions.


\subsubsection*{Law Evaluation Results}
\todo{CA: choose counter-examples and write the derivations to put here.}

Refer \refTab{tab:stm-results} for a summary of the results. 


\newcolumntype{C}{>{\centering\arraybackslash} m{\fill} }  %# New column type

\begin{table}[tb]
\centering
\begin{tabular}[m]{p{.75cm}lCccc}
\hline
 && \multicolumn{2}{c}{\textbf{Lazy}} & \multicolumn{2}{c}{\textbf{Strict}} \\ [0.2ex]
& \textbf{Law}      &|run|&|ex|.&|run|&|ex|.\\ [0.2ex]
\hline
\multirow{4}{*}{\vbox{|Monad| |State|}}
& |PutPut|  &  .  &  .  &  .  &  .  \\
& |PutGet|  &  .  &  .  &  .  &  .  \\
& |GetPut|  &  .  &  .  &  .  &  .  \\
& |GetGet|  &  .  &  .  &  .  &  .  \\[3pt]
%% %
%% \multirow{2}{*}{|Functor|}
%% & |Law1|       &  $\times$  &  $\times$  &  .  &  $\times$  \\
%% & |Law2|       &  .         &  .         &  .  &  .  \\[3pt]
%
\multirow{3}{*}{|Monad|}
& |Law1|       &  .         &  $\times$  &  .  &  $\times$  \\
& |Law2|       &  $\times$  &  $\times$  &  .  &  $\times$  \\
& |Law3|       &  .         &  .         &  .  &  .         \\[3pt]
%% %
%% |Functor| |Monad|
%% & |Law|        &  .  &  .  &  .  &  .  \\
\end{tabular}
\caption{Summary of the Lazy and Strict state monad with
\label{tab:stm-results}
%
  |run| = run equality, |ex|. = exact equality, 
%
  ``$\times$'' = fails QuickCheck test, ``.'' = passes 100 QuickCheck tests.
%
  %% \todo{
  %% The tests were run with ghc version 7.4.2 and the results are same 
  %% both with and without the flag \texttt{-fpedantic-bottoms}.
  %% }
}
\end{table}
